
\paragraph{Documentation and Training} Most Bonsai users are not highly skilled
in ML. We aim at creating extensive documentation with user guides and examples, and delivering
Bonsai courses, so that users can understand Bonsai.ML tools, realize their
power, and use them correctly and effectively to address their research questions.

\paragraph{Community Building} Our experience with Bonsai shows that excellent
functionality and documentation alone are not enough--active engagement with the
community is essential for adoption. We aim at doing this in two ways: (1) by
continuing the Bonsai developers conference, first held in 2024 with over 30
leading neuroscientists and planned again for 2026; and (2) by showcasing
Bonsai.ML’s impact through collaborations with experimental neuroscientists at
top research centres worldwide.

\paragraph{Integrations} Excellent ML packages exist in the open-source
community that would greatly benefit Bonsai users. If integrated into Bonsai.ML, they would greatly enhance Bonsai.ML's reach and help streamline
inference and learning code. We aim at integrating a few of
these packages into Bonsai.ML.

\paragraph{Governance} World-class neuroscientists around the world are heavily
invested in Bonsai. They all want Bonsai to advance for their own benefit and
the benefit of the community. We aim at building a steering committee that will
advise us on the progress of the project and on building a long-term Bonsai.ML
roadmap.
