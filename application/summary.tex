\href{https://bonsai-rx.org/}{Bonsai} is a visual reactive programming language
mostly used for neuroscience experimental control.
%
Bonsai is free and open-source.  Its design emphasises performance,
flexibility, and ease-of-use, allowing scientists with no previous programming
experience to quickly develop their own high-performance data acquisition and
experimental control systems.
%
Bonsai has been adopted in hundreds of laboratories worldwide and has the
largest user base in the systems neuroscience community (7,000 downloads per
year and almost 100 citations per year of the core Bonsai paper
\citep{lopesEtAl15}). In the last year alone, more than 1,000 new users
incorporated Bonsai into their experimental protocols.

The application of
\href{https://www.ukri.org/what-we-do/browse-our-areas-of-investment-and-support/artificial-intelligence-in-bioscience/}{artificial
intelligence in bioscience} is central to UKRI.
%
Specifically, machine learning (ML), the subfield of artificial intelligence
(AI) where systems learn from data, is responsible of all recent breakthroughs
in AI.
%
In 2022 we realised that adding ML functionality to Bonsai would be
transformative for experimental neuroscience.
\href{https://gow.bbsrc.ukri.org/grants/AwardDetails.aspx?FundingReference=BB\%2FW019132\%2F1}{Funded
by BBSRC}, we developed the
\href{https://bonsai-rx.org/machinelearning/}{Bonsai.ML} package, that
integrates into the Bonsai ecosystem state of the art ML methods, like
\href{https://bonsai-rx.org/machinelearning/examples/examples/LinearDynamicalSystems/README.html}{Linear
Dynamical Systems},
\href{https://bonsai-rx.org/machinelearning/examples/examples/HiddenMarkovModels/README.html}{Hidden
Markov Models},
\href{https://bonsai-rx.org/machinelearning/examples/examples/Torch/NeuralNetsTrainedOnline/README.html}{Deep
Neural Networks}, and a
\href{https://bonsai-rx.org/machinelearning/examples/examples/PointProcessDecoder/DecodePositionFromHippocampusSortedUnits/README.html}{Point-Process
Decoder}.

Since the majority of Bonsai users have little exposure to ML, to
maximize impact of Bonsai.ML we need to provide comprehensive documentation.
\textbf{Aim 1 of this proposal is to build such documentation}.

\textbf{Aim 2 is to build a training course on Bonsai and Bonsai.ML}.

Most ML methods in Bonsai.ML are written in Python. They are excellent examples
on how to integrate ML functionality written in other language into Bonsai,
which is critical for our efforts to promote the integration of ML packages
written in Python into Bonsai.ML.
%
However, for time critical applications, implementing ML functionality in C\#
is better.
%
In addition, when adding a new ML model into Bonsai.ML, we need to develop
complex code for learning and inference for the specific model. This code is
generally different for different ML models, leading to heterogeneous code.
%
This code would be much simpler if we used a probabilistic programming language
(PPL) for probabilistic modeling in Bonsai.ML. They make possible to specify
complex probabilistic models in a few lines of code, and they automatically
perform inference and learning in the specified models, exempting users from
the complex, error-prone, and time-consuming task of doing this manually.
%
Since users are exempted from coding inference and learning procedures, PPLs
lead to more homogeneous code across different models.

Thus, using a C\# PPL would lead to faster, simple, more homogeneous and more
extendable inference and learning code in Bonsai.ML.
%
Fortunately, C\# provides an excellent PPL: Infer.NET, develop by Dr.~Tom Minka
and collaborators at Microsoft Research Cambridge. \textbf{Aim~3 is to simplify the
code of all probabilistic models in Bonsai.ML by integrating Infer.NET for
learning and inference}.

To bring together the Bonsai developer community, in December 2024 we organized
the first Bonsai Developers Conference. \textbf{Aim~4 is to foster community building
by organizing the second one in December 2026}.

Because it is a reactive programming language, Bonsai excels in processing
asynchronous events, which is excellent for close-loop control. Currently
Bonsai uses close-loop control only for behavioral experiment.
%
Close-loop control of neural activity is not widely used in the experimental
neuroscience community, mainly because close-loop neural experiments are
challenging to run, and no widely-used experimental platform exist to perform
them.
%
Prof.~Garrett Stanley (Georgia Tech, US), an expert on close-loop neural
control, requested our assistance for the integration into Bonsai.ML of the
software ``\emph{Close Loop Optogenetic Control Tools}'',
\href{https://cloctools.github.io/}{CLOCTools}, that his laboratory had
developed for the control of neural activity with optogenetic inputs.
\textbf{Aim~5 is to integrate CLOCTools into Bonsai.ML}.
%
In doing so we will attract to Bonsai.ML a large community of experimental
neuroscientists doing close-loop control of neural activity.

By the end of the funding period, Bonsai.ML will have excellent documentation,
a better trained user community, a much better code for inference and learning
in probabilistic models (faster, simpler, more homogeneous and extensible), a
more cohesive developers community, and a mature long-term development plan.
%
In addition, expertise in the Bonsai development community will be larger,
thanks to the addition of skills in advanced inference, from the Infer.NET team
at Microsoft Research Cambridge, and skills in close-loop control of neural
activity, from the laboratory of Prof.~Stanley.
