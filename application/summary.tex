\href{https://bonsai-rx.org/}{Bonsai} is a free and open-source visual reactive
programming language widely used for experimental control in neuroscience.
Designed for performance, flexibility, and ease of use, Bonsai enables
scientists with little or no programming background to build high-performance
data acquisition and control systems. With more than 7,000 downloads annually,
nearly 100 citations per year of the core paper, and over 1,000 new users in
2024 alone, Bonsai has established itself as a critical software tool in
systems neuroscience, with adoption continuing to grow worldwide. Its success demonstrates the transformative role that
sustainable, community-driven, open-source research software can play in accelerating
discovery.

A central priority of UKRI is the application of
\href{https://www.ukri.org/what-we-do/browse-our-areas-of-investment-and-support/artificial-intelligence-in-bioscience/}{artificial
intelligence in bioscience}. In 2022 we recognised that integrating machine
learning (ML) into Bonsai could be transformative for experimental
neuroscience. With BBSRC support
(\href{https://gow.bbsrc.ukri.org/grants/AwardDetails.aspx?FundingReference=BB%2FW019132%2F1}{BB/W019132/1}),
we developed \href{https://bonsai-rx.org/machinelearning/}{Bonsai.ML}, which
extends Bonsai with state-of-the-art ML methods. These include
\href{https://bonsai-rx.org/machinelearning/examples/examples/LinearDynamicalSystems/README.html}{Linear
Dynamical Systems},
\href{https://bonsai-rx.org/machinelearning/examples/examples/HiddenMarkovModels/README.html}{Hidden
Markov Models},
\href{https://bonsai-rx.org/machinelearning/examples/examples/Torch/NeuralNetsTrainedOnline/README.html}{Deep
Neural Networks (Torch; trained online)}, and a
\href{https://bonsai-rx.org/machinelearning/examples/examples/PointProcessDecoder/DecodePositionFromHippocampusSortedUnits/README.html}{Point-Process Neural
Decoder}. Embedding these models directly in Bonsai’s reactive programming
environment enables adaptive, data-driven experimental designs that were
previously out of reach for many laboratories.  

To maximise the impact and long-term sustainability of Bonsai.ML, this proposal
pursues six aims:  

\begin{enumerate}

  \item \textbf{Documentation} – Produce comprehensive, user-centred
documentation that makes ML tools accessible to non-specialists across the
neuroscience community.  

  \item \textbf{Training} – Develop and deliver a practical training course on
Bonsai and Bonsai.ML, building community and lowering barriers to adoption.  

  \item \textbf{Maintainability} – In collaboration with Microsoft Research
  Cambridge, integrate their C\# probabilistic programming library
  \emph{Infer.NET} into Bonsai.ML. This will reduce and unify inference and
  learning code, making it faster, more maintainable, and more extensible,
  while embedding the expertise of a world-leading industrial research group
  into the Bonsai ecosystem.

  \item \textbf{Community reach} – Engage neuroscientists interested in
\emph{closed-loop} neural experimentation by supporting the integration of
\href{https://cloctools.github.io/}{CLOCTools} within Bonsai.ML, in
collaboration with Prof.~Garrett Stanley (Georgia Tech). This partnership will
attract a new community of researchers to Bonsai and broaden its reach to an
emerging but underrepresented area of neuroscience, namely closed-loop neural control.

  \item \textbf{Community building} – Strengthen the developer community by
organising the second Bonsai Developers Conference in December 2026, building
on the successful inaugural event in 2024.  

  \item \textbf{Governance} – Establish a steering committee to guide the
long-term vision, prioritise sustainability, and supervise development
throughout the grant.

\end{enumerate}

By the end of the funding period, Bonsai.ML will provide high-quality
documentation and training resources, a robust and maintainable codebase, and a
stronger developer community with expanded expertise and broader reach. In
particular, collaborations with Microsoft Research Cambridge and
Prof.~Garrett Stanley’s laboratory will embed cutting-edge knowledge in
probabilistic inference and closed-loop experimentation. Together, these
efforts will ensure Bonsai remains a robust, widely adopted research
software platform, aligned with UKRI’s strategic priorities in artificial
intelligence, bioscience, and the effective maintenance of research software infrastructure.

