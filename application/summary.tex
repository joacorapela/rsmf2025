\href{https://bonsai-rx.org/}{Bonsai} is a visual reactive programming language
mostly used for neuroscience experimental control.
%
Bonsai is free and open-source.  Its design emphasises performance,
flexibility, and ease-of-use, allowing scientists with no previous programming
experience to quickly develop their own high-performance data acquisition and
experimental control systems.
%
Bonsai has been adopted in hundreds of laboratories worldwide and has the
largest user base in the systems neuroscience community (7,000 downloads per
year and almost 100 citations per year of the core Bonsai paper
\citep{lopesEtAl15}). In the last year alone, more than 1,000 new users
incorporated Bonsai into their experimental protocols.

The application of
\href{https://www.ukri.org/what-we-do/browse-our-areas-of-investment-and-support/artificial-intelligence-in-bioscience/}{artificial
intelligence in bioscience} is central to UKRI.
%
Specifically, machine learning (ML), the subfield of artificial intelligence
(AI) where systems learn from data, is responsible of all recent breakthroughs
in AI.
%
In 2022 we realised that adding ML functionality to Bonsai would be
transformative for experimental neuroscience.
\href{https://gow.bbsrc.ukri.org/grants/AwardDetails.aspx?FundingReference=BB\%2FW019132\%2F1}{Funded
by BBSRC}, we developed the
\href{https://bonsai-rx.org/machinelearning/}{Bonsai.ML} package, that
integrates into the Bonsai ecosystem state of the art ML methods, like
\href{https://bonsai-rx.org/machinelearning/examples/examples/LinearDynamicalSystems/README.html}{Linear
Dynamical Systems},
\href{https://bonsai-rx.org/machinelearning/examples/examples/HiddenMarkovModels/README.html}{Hidden
Markov Models},
\href{https://bonsai-rx.org/machinelearning/examples/examples/Torch/NeuralNetsTrainedOnline/README.html}{Deep
Neural Networks}, and a
\href{https://bonsai-rx.org/machinelearning/examples/examples/PointProcessDecoder/DecodePositionFromHippocampusSortedUnits/README.html}{Point-Process
Decoder}.

Since the majority of Bonsai users have little exposure to ML, to
maximize impact of Bonsai.ML we need to provide comprehensive documentation.
\textbf{Aim 1 of this proposal is to build such documentation}.

\textbf{Aim 2 is to build a training course on Bonsai and Bonsai.ML}.

Most ML methods in Bonsai.ML are written in Python. They are excellent examples
on how to integrate ML functionality written in other language into Bonsai.
%
However, for time critical applications, implementing ML functionality in C\#
is better.
%
In addition, the inference and learning code in Bonsai.ML would be much
simpler, more homogeneous, and applicable to a broader set of moels, if we used
a probabilistic programming language (PPL) in Bonsai.ML.
%
Fortunately, C\# provides an excellent PPL, Infer.NET, develop by Dr.~Tom Minka
and collaborators at Microsoft Research Cambridge. \textbf{Aim~3 is to improve
the maintainability of code of all probabilistic models in Bonsai.ML by
integrating Infer.NET for learning and inference}.

To bring together the Bonsai developer community, in December 2024 we organized
the first Bonsai Developers Conference. \textbf{Aim~4 is to foster community
building by organizing the second Bonsai Developers Conference in December
2026}.

Because it is a reactive programming language, Bonsai excels in processing
asynchronous events, which is excellent for close-loop control. Currently
Bonsai uses close-loop control only for behavioral experiment.
%
Despite its strong potential, close-loop control of neural activity is not
widely used by the experimental neuroscience community, mainly because
close-loop neural experiments are challenging to run, and not widely-used
experimental platform exist to perform them.
%
Prof.~Garrett Stanley (Georgia Tech, US), an expert on close-loop neural
control, requested our assistance for the integration into Bonsai.ML of the
software ``\emph{Close Loop Optogenetic Control Tools}'',
\href{https://cloctools.github.io/}{CLOCTools}, that his laboratory had
developed. \textbf{Aim~5 is to integrate CLOCTools into Bonsai.ML}.
%
In doing so we will attract to Bonsai.ML a large community of experimental
neuroscientists interested in doing close-loop control of neural activity.

\textbf{Aim~6 is to assemble and meet with a steering committee that will guide
us in building a long-term roadmap, and supervise the development of this
grant}.

By the end of the funding period, Bonsai.ML will have excellent documentation,
a better trained user community, a much better code for inference and learning
in probabilistic models (faster, simpler, more homogeneous and extensible), a
more cohesive developers community, and a mature long-term development plan.
%
In addition, expertise in the Bonsai development community will be larger,
thanks to the addition of skills in advanced inference, from the Infer.NET team
at Microsoft Research Cambridge, and skills in close-loop control of neural
activity, from the laboratory of Prof.~Stanley.
