
\documentclass[12pt]{article}

\usepackage{xcolor}
\usepackage{tcolorbox}
\usepackage[colorlinks=true]{hyperref}
\usepackage{verbatim}
\usepackage[margin=1.5cm]{geometry}
\usepackage{natbib}

\newenvironment{instruction}{%
    \begin{tcolorbox}[colback=red!5,colframe=red,title=Instruction]%
}{%
    \end{tcolorbox}%
}

\title{Consolidating Bonsai as a Standard for Neuroscience Intelligent
Experimental Control}

\author{Joaqu\'{i}n Rapela}

\begin{document}

\maketitle

\section*{Summary}

\begin{instruction}

Please provide a summary in plain English of your proposed work.

This summary will be made publicly available on external-facing websites, therefore do not include any confidential or sensitive information.

Please note that by submitting an application, you consent to the information provided on this page (Project Summary, UKRI areas) being publicly disseminated. This includes information from both successful and unsuccessful applications. Information supplied in the other sections of the application will not be published.

It may also be used to help identify suitable reviewers.

\end{instruction}

\subsection*{Summary (500 words)}

\pagebreak

\section*{Core Team}

\begin{instruction}

Tell us who will deliver the proposed work. Create an entry for each core team member.

This section is for naming the people who will make key contributions to the work.

Any unnamed role resource, such as a research software engineer role where an individual has not yet been identified, should be included and justified in the Resources section. All project leads (including international ones, as well as co-leads) need to be named.

If a single person has multiple roles, select the main one under "Role"
There can only be one project lead, and they must be based at an eligible UK Research Organisation. The project lead should normally be the person submitting the application.
If the project lead has changed since the EoI, please contact us to have an invitation to the new lead issued.

\end{instruction}

\pagebreak

\section*{Software}

\begin{instruction}

Please provide details of the software to be supported.

This information may be used by reviewers as part of the assessment of your application. It may also be used to provide an anonymised and aggregated summary across all applications.

\end{instruction}


\begin{description}

    \item[a. Software to be maintained:] \href{https://bonsai-rx.org/}{Bonsai}

    \item[b. Code repository (optional):] \url{https://github.com/bonsai-rx/bonsai}

    \item[c. Website (optional):] \url{https://bonsai-rx.org/}

    \item[d. Year development started:] 2012

    \item[e. Year of first release:] 2014

    \item[f. Programming language:] C\#

\end{description}



\pagebreak

\section*{Vision}

\begin{instruction}

Please describe what you are hoping to achieve with this funding.

Discuss both your vision and objectives for the technical development, as well as your plans to improve your software's sustainability and support EDIA considerations.

\end{instruction}

\subsection*{Vision (300 words)}

\begin{instruction}

Explain how your proposed work will:

\begin{itemize}
    \item have a benefit and impact on research, with specific examples of research impact in the UK
    \item build the community around the software
    \item increase the adoption of the software
    \item improve the maintainability of the software
    \item advance good practice
\end{itemize}

\end{instruction}

More than ten years ago our business partner, Dr.~Goncalo Lopes, created
\href{https://bonsai-rx.org/}{Bonsai}, a reactive visual programming
language most widely used for neuroscience experiment control.
%
Bonsai has been adopted by thousands of users in the UK and all around the
world (7,000 downloads per year and 1,000 citations per year of the core Bonsai
paper~\citep{lopesEtAl15}).

In 2022, we recognized that integrating ML into the Bonsai platform could
enable a radically new form of intelligent experimental control.
%
To realise this vision, sponsored by
\href{https://gow.bbsrc.ukri.org/grants/AwardDetails.aspx?FundingReference=BB\%2FW019132\%2F1}{BBSRC},
we developed \href{https://bonsai-rx.org/machinelearning}{Bonsai.ML}, a package
that brings ML functionality into the Bonsai ecosystem.

Bonsai.ML aims to bridge the gap between cutting-edge machine learning and
experimental neuroscience by providing accessible tools that help researchers
integrate ML into their workflows and advance our understanding of brain
function.

The proposed documentation, training and dissemination activities will help
users of Bonsai.ML better understand the ML methods in the package, perform
more sophisticated neuroscience experiments, and produce unprecedented new
neuroscientific findings.
%
The proposed integration activities will equip Bonsai.ML with state-of-the-art
methods for closed-loop control of neural activity, while adding a
probabilistic programming language to the Bonsai ecosystem that will
accelerate, standardize, and simplify inference, resulting in more reliable,
less error-prone, and easier-to-maintain code.

These activities will attract to Bonsai new experimental neuroscience users
interested in adding ML functionality to their experiments, and create a new
Bonsai community of ML methods developers interested in applying their methods
in close loop to the real-time data generated by Bonsai.

Bonsai has already transformed experimental design, allowing researchers to
perform studies of complexity previously unimaginable. With Bonsai.ML, we aim
to spark a second revolution — empowering scientists to control experiments in
intelligent, adaptive ways they had never thought possible. The activities
proposed here are critical to achieve this goal.



\subsection*{Objectives (200 words)}

\begin{instruction}

Clearly describe the aims and objectives of this work.

\end{instruction}

\subsection*{Timeliness and Sustainability (300 words)}

\begin{instruction}

Justify why it is important that your project is funded in this round, and include:

    \begin{itemize}
        \item why it is important that this work is funded now, rather than at a different time
        \item what other funding sources have been investigated, and why these are not suitable
        \item how the software will be sustained and maintained after the RSMF funding ends, and how the RSMF funding will help achieve this
    \end{itemize}

\end{instruction}

\subsection*{EDIA (300 words)}

\begin{instruction}
Explain how you will embed equity, diversity, inclusivity and accessibility considerations into your proposed work and the software being maintained, and how these will guide your aims, objectives, activities and outputs.

This can include (but is not limited to) considerations applying to your team, your community, and your software.
\end{instruction}

\pagebreak

\section*{Approach (2000 words)}

\subsection*{Approach}

\begin{instruction}

Explain what work you have planned, and how you will manage this. Cover both the technical as well as project management areas.

Planned work

    \begin{itemize}
        \item where applicable: summary of any previous work and how this will be built upon and progressed
        \item where applicable: a clear and transparent methodology
        \item effective and appropriate activities to achieve your objectives
        \item which team members are responsible for each activity and task
        \item expected outputs for each task and how they relate to each other
        \item a timeline with a feasible workplan
        \item strategy to maximise translation of outputs into outcomes and impacts
    \end{itemize}

Management

    \begin{itemize}
        \item how the work will be managed and progress monitored and evaluated
        \item how the current software project governance and infrastructure will contribute to the success of the work
        \item what risks to delivery were identified and how they will be managed
    \end{itemize}


\end{instruction}


\begin{figure}
    \centering
    \includegraphics[width=6in]{activitiesGraphs/activities_larger.png}

    \caption{Proposed activities. The initials below each task name indicate
    the responsible team member (JR: Joaquin Rapela, NG: NeuroGEARS Ltd). See
    text for details.}

\end{figure}

\subsection*{Approach}

\subsubsection*{Documentation}

\paragraph{Summary:} Expand Bonsai.ML documentation.

\paragraph{Previous work:} Bonsai.ML already contains detailed
\href{https://bonsai-rx.org/machinelearning/index.html}{documentation} for all
distributed packages. This documentation includes installation instructions,
API listings, and Bonsai working examples that users can copy and paste into
their Bonsai local runtime and execute. We will improve this documentation as
described next.

\paragraph{Future work (responsible team member):}\mbox{}\\
\begin{itemize}

    \item \textbf{Re-structure documentation following the style of
        \href{https://scikit-learn.org/}{scikit-learn} (NG)}. Our current
        documentation lacks accessible statistical explanations of each method
        and clear links between related methods.  We will adopt the style of
        \href{https://scikit-learn.org/}{scikit-learn}, where methods are
        introduced with high-level descriptions, grouped into categories, and
        cross-referenced to practical examples.

    \item\textbf{Provide more examples (NG)} on the application of the already
        integrated ML methods to new types of behavioural and neural data.
        %
        These examples will be similar to the
        \href{https://bonsai-rx.org/machinelearning/examples/README.html}{existing
        ones}, but will be more detailed.
        %
        They should attract to Bonsai.ML a wider group of experimental
        neuroscientists.

    \item\textbf{Include video tutorials (NG)} on the use of the Bonsai.ML
        package, as done in
        \href{https://mark-kramer.github.io/Case-Studies-Python}{Case Studies
        for Neural Data Analysis}, for example
        \href{https://youtu.be/Oj9e2bB3BfI}{here}.

    \item\textbf{Add documentation for methods developers (NG)} as the current
        one is targeted to Bonsai.ML users. For example, we will provide
        detailed instructions to Python developers on how to integrate their
        methods into Bonsai.ML, including explanations about how handles
        Python's global interpreter lock in C\#.

    \item\textbf{Create case studies (JR)} for intelligent experimental control
        in Bonsai, similar to the
        \href{https://mark-kramer.github.io/Case-Studies-Python/intro.html}{case
        studies for neural data analysis in Python}, or those for
        \href{https://mbmlbook.com/index.html}{model based machine learning},
        by our Microsoft collaborators.

    \item\textbf{Publish a first Bonsai.ML paper (JR)} describing its functionality, as
        companion papers substantially increases the adoption of software
        packages \citep{lopesEtAl15,guilbeaultEtAl21}.

\end{itemize}

\noindent\rule{\textwidth}{1pt}
\subsubsection*{Training}
\paragraph{Summary:} Organise a Bonsai course with a dedicated Bonsai.ML module.

\paragraph{Previous work:} Since 2017, NeuroGEARS Ltd has organised at least
two Bonsai courses per year at different universities, and deliver a
one-week-long Bonsai course, hosted at the Sainsbury Wellcome Centre, for
approximately 20 students, targeted to Bonsai users with intermediate
understanding of the language. The structure of the course will be similar to
previous ones (e.g., \href{https://neurogears.org/st-andrews-2024/}{2024 Bonsai
Course at St.~Andrews University}) and is outlined briefly below.

\begin{description}
    \item[Day 1 - Introduction to Bonsai:] What is visual reactive programming? Introduction to Marble diagrams and how to read them. Learning your way around the Bonsai IDE. How to measure and control almost anything with Bonsai. Real-time tracking of colored objects, moving objects and contrasting objects. Measuring behavior using voltages and an Arduino.

    \item[Day 2 - Real-time closed-loop experiments:] Fundamental reactive operators. Data synchronization and measuring closed-loop latency. Conditional effects: triggering a stimulus based on video activity. Continuous feedback: modulate stimulus intensity with speed or distance. Continuous and conditional feedback: closed-loop experiment building blocks. Synchronising asynchronous data streams.

    \item[Day 3 - Operant behavioral tasks:] Sharing observable sequences. Creating dynamic observable sequences with higher-order operators. Modeling trial sequences: states, events, and side-effects. Driving state transitions with external inputs. Best practices for composing complex workflows.

    \item[Day 4 - Machine learning:] Basics of online probabilistic machine learning. Online Bayesian linear regression. Interfacing Bonsai with Python. Linear dynamical systems. Hidden Markov Models. Neural decoding models. Neural latents models.

    \item[Day 5 - Best practices:] How to extend Bonsai with scripting. Reproducible deployment and versioning of experiments. Bonsai hackathon and project presentations. Closing remarks.
\end{description}

\paragraph{Summary:} Organise Bonsai course, with a Bonsai.ML module.

\paragraph{Previous work:} Since 2017, NeuroGEARS Ltd has organised at least
two Bonsai courses per year at different universities, and
\href{https://bonsai-rx.org/learn/}{some of them} can be viewed online. We will
deliver a one-week-long Bonsai course, hosted at the Sainsbury Wellcome Centre,
for approximately 20 students, targeted to Bonsai users with intermediate
understanding of the language. The structure of the course will be similar to
previous ones (e.g., \href{https://neurogears.org/st-andrews-2024/}{2024 Bonsai
Course at St.~Andrews University}).

\paragraph{Outputs:} course delivery, online course material (including lecture
slides, worksheets and video recordings).

\paragraph{Responsible team members:} GL, JR.

\noindent\rule{\textwidth}{1pt}
\subsubsection*{Integrations}

\subsubsection*{Probabilistic programming: Infer.NET}

\paragraph{Background:} Most of the probabilistic models currently integrated
into Bonsai are implemented in Python. These serve as excellent demonstrations
of how Python applications can connect to the Bonsai ecosystem, and they are
central to our aim of attracting Python developers to contribute to Bonsai.ML.
%
However, Python implementations are substantially slower than equivalent C\#
code. For demanding real-time applications, C\# implementations are preferable,
especially when expressed in a probabilistic programming language (PPL).

In addition, the existing Python and C\# implementations of probabilistic
models (e.g., linear dynamical systems, hidden Markov models, Bayesian linear
regression — see
\href{https://bonsai-rx.org/machinelearning/examples/README.html}{examples})
are relatively complex and heterogeneous, in the sense that the implementation
of learning and inference in linear dynamical systems is non-trivial and has
little in common with the implementation of learning and inference in Bayesian
linear regression or in hidden Markov models.
%
In contrast, implementations in a C\# PPL would be much simpler, since PPLs
abstract from their users the complexities of their inference algorithms,
leading to sophisticated inferential methods implemented in a few lines of
code.
%
Also, thanks to this abstraction, implementations of different probabilistic
models in PPLs are substantially more homogeneous in PPLs than in general
purpose ones.

Currently, when we decide to incorporate a new probabilistic model into
Bonsai.ML, we need to implement the learning and inference algorithms for the
specific model, which is generally quite complex and time consuming.
%
In contrast, implementing a new probabilistic model in a PPL only requires the
specification of how the model generates observations, without the
need of specific learning or inference algorithms.
%
Thus, integrating a PPL into Bonsai.ML would greatly simplify the addition of
new probabilistic models to the language.

Fortunately, C\# has an excellent PPL:
\href{https://dotnet.github.io/infer/}{Infer.NET}, developed at Microsoft
Research Cambridge since 2004, used in
\href{https://dotnet.github.io/infer/papers.html}{hundreds of papers}, and
\href{https://www.microsoft.com/en-us/research/blog/the-microsoft-infer-net-machine-learning-framework-goes-open-source/}{open-sourced
in 2018}.  Infer.NET uses deterministic approximate inference, enabling fast
and scalable solutions. For
\href{https://www.microsoft.com/en-us/research/blog/the-microsoft-infer-net-machine-learning-framework-goes-open-source/}{example},
it has powered systems that extract knowledge from billions of web pages
(petabyte-scale data) — the kind of scalability critical for real-time
inference in Bonsai.

\paragraph{Previous work:} Our Microsoft collaborator, Dr.~Tom Minka, invented
a seminal algorithm for inference in graphical models, the Expectation
Propagation algorithm~\citep{minka01}, and is the lead developer of Infer.NET.
Please refer to his letter of support.

\paragraph{Tasks:}\mbox{}\\

\begin{description}

    \item[in\_learn:] the responsible team member is skilled in probabilistic
        programming, but not in Infer.NET. He will invest two weeks in learning
        well the language.

    \item[in\_OBLR:]  Implement in Infer.NET the Online Bayesian Linear
        Regression model, currently implemented in C\# in Bonsai.ML. Develop
        test cases to check that the output of the Infer.Net and previous C\#
        implementations are equal.

    \item[in\_LDS:] Implement in Infer.NET the Linear Dynamical Systems model,
        currently implemented in Python. Develop test cases to check that the
        output of the Infer.Net and previous Python implementations are equal.

    \item[in\_HMM:] Implement in Infer.NET the Hidden Markov Model, currently
        implemented in Python. Develop test cases to check that the output of
        the Infer.Net and previous Python implementations are equal.

    \item[in\_PPdecoder:] Implement in Infer.NET the Point Process Decoding
        model, currently implemented in Python. Develop test cases to check
        that the output of the Infer.Net and previous Python implementations
        are equal.

    \item[in\_docs:] For each of the models integrated above, we will add
        extensive documentation explaining how the models were implemented in
        Infer.NET. The aim is to enable Bonsai.ML users to learn from these
        examples and apply the same principles to build and perform inference
        on their own custom probabilistic models, beyond the ones provided in
        Bonsai.ML-Infer.NET.

\end{description}

\paragraph{Impact:}
This integration will accelerate
inference, simplify and standardise inference programs, and empower Bonsai
users to create new probabilistic models and inference algorithms with just a
few lines of code. Consequently, this activity will drastically improve the
maintainability of the software and facilitate the incorporation of new
probabilistic functionality to the Bonsai ecosystem.

\paragraph{Milestones and Indicators:}\mbox{}\\

\begin{description}

    \item[in\_m1:] OBLR implemented in Bonsai.ML.Infer.NET.

    \item[in\_i1:] Package Bonsai.ML.Infer.NET.OBLR published in nuget.org.

    \item[in\_m2:] LDS implemented in Bonsai.ML.Infer.NET.

    \item[in\_i2:] Package Bonsai.ML.Infer.NET.LDS published in nuget.org.

    \item[in\_m3:] HMM implemented in Bonsai.ML.Infer.NET.

    \item[in\_i3:] Package Bonsai.ML.Infer.NET.HMM published in nuget.org.

    \item[in\_m4:] Point Process Decoder implemented in Bonsai.ML.Infer.NET

    \item[in\_i4:] Package Bonsai.ML.Infer.NET.PP.decoder published in
        nuget.org.

    \item[in\_m5:] Detailed documentation added to all nuget packages above.

    \item[in\_i5:] Eetailed documentation available in the above nuget
        packages.

    \item[in\_m6:] Bonsai.ML.Infer.NET.* packages published.
    \item[in\_i6:] Packages Bonsai.ML.Infer.NET.* available in nuget.org.

\end{description}

\paragraph{Responsible team member:} JR.

\subsubsection*{Closed-loop optogenetic control tools: CLOCTools}

\paragraph{Background:} Closed-loop neural control represents a transformative
advance in neuroscience:  rather than delivering stimulation at fixed,
open-loop schedules, it enables precisely timed interventions based on ongoing
brain and behavioural activity. This paradigm allows researchers to move beyond
observing correlations to  directly testing causal mechanisms of neural
dynamics, plasticity, and behaviour. Critically, closed-loop stimulation has
already proved transformative in the  clinic, for example in deep brain
stimulation for Parkinson’s disease and  epilepsy, while its extension to other
disorders (e.g. depression, obsessive  compulsive disorder) remains an active
area of research. Despite this promise, widespread adoption has been limited
by the lack of accessible, well-engineered,  and sustainable software
frameworks for real-time experimental control.

Prof.~Garrett Stanley (Georgia Tech and Emory University, US) is a pioneer in
closed-loop neuroscience, having developed groundbreaking methods that combine
real-time neural control with systems neuroscience. He recently contacted us to
explore integrating their existing
\href{https://cloctools.github.io/}{CLOCTools}, originally implemented in
RTXI/C++, into Bonsai.ML.  This represents a unique opportunity: Bonsai already
excels at real-time closed-loop control in behavioural experiments, and
extending it to include state-of-the-art closed-loop neural control will
position the platform as the first sustainable, general-purpose framework for
both levels of experimentation.

An important reason for the poor uptake of closed-loop methods in neuroscience
could be that existing implementations are often ad hoc, difficult to install
or  extend, and not integrated with software for experimental control. By
providing Bonsai users with accessible, open-source, and sustainably engineered
tools for closed-loop control, we will directly address this barrier and enable
a new type of experimentation where researchers can both read and write the
neural code in real time. This will accelerate discovery across both basic and
translational neuroscience.

\paragraph{Previous work:} Members of Prof.~Stanley’s lab have already
prototyped some of their closed-loop control methods in Bonsai using our
\href{https://bonsai-rx.org/python-scripting/}{Python scripting interface}.
Please refer to the repositories in the
\href{https://github.com/ndac-bonsai}{ndcac-bonsai} organisation containing
these prototypes (ndac stands for neural dynamic adaptive control).

\paragraph{Tasks:}\mbox{}\\

\begin{description}

    \item[hsc\_C\#\_Infer.NET:] Implement the Python package Hybrid System
        Control in C\#.
	%
        Runtime performance is critical for the control of neural system, and a
        C\# implementation should be much faster than a Python one.
	%
        The Python implementation uses the ssm Python library to perform
        inference in linear dynamical systems. Instead, we will
        use Bonsai.ML.Infer.NET.LDS previously developed. (4 weeks)

    \item[hsc\_test\_cases:] Develop test cases for the C\# implementation of
        HybridSystemControl, comparing is functionality with that of the
        original Python implementation. Fix any resulting problem. (2 weeks)

    \item[hsc\_bonsai\_integration:] Integrate the C\# implementation of HybridSystemControl
        into Bonsai, generating the Bonsai.ML.HybridSystemControl package.  (2
        weeks)

    \item[hsc\_eval\_synthetic:] Evaluate the package
        Bonsai.ML.HybridSystemControl with synthetic data and fix problems.
        These evaluation will focus on testing if the package can achieve the
        real-time latency constraints required in neuroscience applications. (2
        weeks)

    \item[hsc\_eval\_exp:] Assist the laboratory of Prof.~Stanely in
        testing Bonsai.ML.HybridSystemControl with experimental data, and
        address problems. (4 weeks)

    \item[hsc\_docs:] Add documentation to Bonsai.ML.HybridSystemControl, in a
        similar style as the new documentation of other Bonsai.ML packages. (3
        weeks)

    \item[hsc\_release:] Release the Bonsai.ML.HybridSystemControl package. (1
        week)

\end{description}

\paragraph{Milestones and Indicators:}\mbox{}\\

\begin{description}

    \item[hsc\_m1:] Package HybridSystemControl implemented in C\#.

    \item[hsc\_i1:] Package HybridSystemControl published in nuget.org.

    \item[hsc\_m2:] Test cases added  to HybridSystemControl.

    \item[hsc\_i2:] Test cases available in nuget package HybridSystemControl.

    \item[hsc\_m3:] Package Bonsai.ML.HybridSystemControl created.

    \item[hsc\_i3:] Package Bonsai.ML.HybridSystemControl available in GitHub
        repository.

    \item[hsc\_m4:] Package Bonsai.ML.HybridSystemControl evaluated with
        synthetic data.

    \item[hsc\_i4:] Workflows demonstrating the correct functionality with
        syntetic data of the package Bonsai.ML.HybridSystemControl available in
        GitHub repository.

    \item[hsc\_m5:] Functionality of Bonsai.ML.HybridSystemControl demonstrated
        with experimental data.

    \item[hsc\_i5:] Experimental evidence available in GitHub repository demonstrating that the package
        Bonsai.ML.HybridSystemControl succeeded holding the firing rate of a
        single neuron in the thalamus at a steady level with optogenetic
        inputs, while a mouse was visually stimulated.

    \item[hsc\_m6:] Documentation added to Bonsai.ML.HybridSystemControl.

    \item[hsc\_i6:] API documentation, examples and tutorials available in
        GitHub repository.

    \item[hsc\_m7:] Package Bonsai.ML.HybridSystemControl released.

    \item[hsc\_i7:] Bonsai.ML.HybridSystemControl available in nuget.org.

\end{description}

\paragraph{Responsible team members:} JR

\noindent\rule{\textwidth}{1pt}
\subsubsection*{Governance}

We will create a Bonsai.ML steering committee that will be responsible for
approving project milestones and advise us on building a long-term development
roadmap for Bonsai.ML.
%
Responding to guidance and feedback from the steering committee ensures
Bonsai.ML addressed pressing neuroscience needs on an international scale.

Several renowned experimental and computational neuroscientists around the
world are heavily invested in Bonsai, are very interested in adding ML
functionality to their Bonsai workflows, and have agreed to join the Bonsai.ML
steering committee. We list them below.

\begin{description}

    \item[Prof.~Garrett Stanley,] leader of the Laboratory for the Control of
        Neural Systems, Georgia Tech, US. Expert on close-loop control of
        neurophysiological systems. He contacted us to migrate to Bonsai
        a package for real-time cortical state estimation and the CLOCTools
        referred above, both originally written in C++/RTXI.

    \item[Prof.~Aman Saleem,] director of the Saleem Lab at the Institute for
        Behavioural Neuroscience, University College London. Prof.~Saleem is the
        author of \href{https://bonvision.github.io/}{Bon-Vision}, a software
        package that creates and controls visual environments in close loop,
        built on top of Bonsai.

    \item[Prof.~Josh Siegle,] senior scientist and lead of the
        electrophysiology group at the Allen Institute for Neural Dynamics. He
        leads the development of Open Ephys GUI, which is tightly integrated
        with Bonsai.

    \item[Prof.~Ken Harris,] co-director of the Cortexlab, University College
        London, and founding director of the International Brain Laboratory,
        that uses Bonsai for reproducible experimental control across 22
        laboratories around the world.

    \item[Prof.~Athena Akrami,] director of the Learning, Inference and Memory
        Lab, at the SWC, that uses Bonsai extensively for the control of
        complex rodent experiments in her lab.

\end{description}

\noindent\rule{\textwidth}{1pt}
\subsection*{Management}

We will conduct management activities at different frequencies:

\begin{description}

    \item[Twice a year:] we will convene the steering committee. Two weeks in
        advance of each meeting, we will circulate a progress report
        summarising achieved milestones, proposed future activities, and the
        most recent version of the Bonsai.ML long-term roadmap.
        %
        The steering committee will review the report, endorse completed
        milestones, and provide feedback or suggestions on planned activities.
        %
        Based on project progress and committee input, we will revise the
        long-term roadmap, ensuring that Bonsai.ML development remains aligned
        with community needs and strategic goals.
        %
        Minutes and key decisions from these meetings will be documented and,
        where possible, shared with the Research Software Maintenance Fund
        through appropriate reporting channels.

    \item[Every month:] we will hold meetings between the project lead, the
        project co-lead, the RSE, and the external project co-lead, to evaluate
        the project progress.

    \item[Weekly:] as has been our practice since the start of the Bonsai.ML
        project, the RSEs will meet with the external project co-lead to
        discuss issues that appeared during the week, review activities for the
        following week, and adjust project directions.

\end{description}

Meetings with collaborators will be arranged as needed.
%
At the SWC, GCNU and NG we are experimental and computational neuroscientists
with successful collaborative experience, and we have no doubt that the
proposed collaborations will be of the same kind,
%
specially since we have successfully interacted in the past with most of the
propose collaborators.
%
Please refer to their letters of support.


\subsection*{Workplan (optional)}

\begin{instruction}
You may choose to upload a one page document such as a Gantt chart to help illustrate timing of and links among the activities.
\end{instruction}

\pagebreak

\section*{Capability to deliver (1000 words)}

\begin{instruction}
Provide evidence of how you and your team have:

    \begin{itemize}
        \item the relevant experience (appropriate to career stage) to deliver the objectives
        \item the right balance of skills and expertise to cover the proposed work
        \item the appropriate leadership and management skills to ensure delivery
        \item contributed to developing good practice in your communities
    \end{itemize}

Where applicable, discuss your approach to developing others.

\end{instruction}

\pagebreak

\section*{Project partners}

\begin{instruction}

A project partner is a collaborating organisation that plays an integral role in the proposed work. You should describe the nature of this support in the Approach section of your application.

Project partners contribute to the delivery of the project and should not normally request funding from the grant. However, travel and subsistence costs incurred by the lead organisation to enable project partner involvement may be included—these must be fully justified in the Resources section.

You cannot include an individual as an applicant (i.e., project lead, co-lead, or any Core Team role) if they, or their organisation, are named as the project partner contact.

All project partners must be listed as contributors below, and a letter of support for each must be uploaded as a single combined PDF.

\end{instruction}

\subsection*{Letter(s) of Support (optional)}

\pagebreak

\section*{Resources}

\begin{instruction}

Please provide details of the funding requested.

Use the breakdown categories listed in this section, and discuss the main resource requirements.

You will also need to upload a consolidated budget from the lead organisation using a standard FEC costing format.

\end{instruction}

\subsection*{Justification of resources (1000 words)}

\begin{instruction}

Justify the application’s more costly resources, in particular:

\begin{itemize}
    \item any staff costs
    \item significant costs related to collaboration or community engagement
    \item any consumables beyond typical requirements
    \item infrastructure costs
    \item all resources that have been costed as ‘Exceptions’
\end{itemize}

You do not need to justify Estates and Indirect costs.

We are not looking for a detailed breakdown of each cost, but want you to demonstrate how the resources you are applying for are comprehensive, appropriate and justified, and represent the optimal use of resources to achieve the intended outcomes.

\end{instruction}

\subsubsection*{Total funding requested}

\subsubsection*{Directly incurred - Staff}

\subsubsection*{Directly incurred - Travel and Subsistence}

\subsubsection*{Directly incurred - Other}

\subsubsection*{Directly allocated - Staff}

\subsubsection*{Directly allocated - Estates}

\subsubsection*{Directly allocated - Other}

\subsubsection*{Indirects}

\subsubsection*{Exceptions}

\subsubsection*{Budget}

The lead organisation's consolidated budget in a standard FEC costing format.

\bibliographystyle{apalike}
\bibliography{bonsai,longDurationExperimentation}

\end{document}
