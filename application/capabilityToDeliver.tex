
This proposal is mutli-disciplinary, requiring skills on software engineering,
machine learning and both experimental and computational neruroscience. Our
team has these skills at the leadership and developers level, as we describe
next.

\subsection*{Relevant Experience}

\paragraph{Prof.~Tom Mrsic Flogel} (project lead) is an experimental
neuroscientists and director of the SWC. Research in his laboratory aims to explain how the
brain makes decisions by combining sensory information with previously learned
knowledge. This research relies on training mice to perform complex behavioural
tasks, and requires complex experimental design, combined with cutting-edge
recording and data-analytic techniques, in order to produce shareable code and
datasets (e.g.
\href{http://mouse.vision/han2017/}{http://mouse.vision/han2017/}) and
publications (Orsolic et al 2021 Neuron; Clancy et al 2019 Nature Neuroscience;
Chabrol et al 2019 Neuron; Han et al 2018 Nature; Kim et al 2018 Neuron; Khan
et al 2018 Nature Neuroscience). His research team has also established brain-machine
interfaces, wherein brain activity is used to control external actuators,
requiring closed-loop experimental control (Clancy and Mrsic-Flogel, 2021
Neuron).  As the behavioural tasks used in his lab require complex
software-control of data acquisition and data analysis pipelines, he knows
first-hand their crucial importance for driving and enabling neuroscientific
research.

He has published 49 peer-reviewed papers, with an h-index of 35 (calculated by
google scholar).  He is a founding member of the International Brain
Laboratory (IBL), a global consortium which brings together researchers from
experimental and theoretical disciplines to understand decision-making in the
in brain (IBL et al 2019 Neuron). The Bonsai ecosystem is critical to IBL, as
it ensures that experimental control, stimulus presentation and data
acquisition can be identically reproduced across all participating labs in UK,
Europe and USA (IBL et al 2021, eLife)

The expertise of Prof.~Mrsic Flogel will be critical to supervise the
neuroscientific aspects of the proposed documentation, collaborations and
integrations.

\paragraph{Prof.~Maneesh Sahani} (project co-lead) is a computational
neuroscientists and director of the Gatsby Unit. He has authored over 100
peer-reviewed scientific papers, with an h-index (computed by Google scholar)
of 47. A substantial component of his research focuses on the development of
advance machine-learning based tools for neuroscience research. This included
his PhD research. His thesis \textit{Latent Variable Models for Neural Data
Analysis} introduced new techniques of probabilistic inference and applied
these to the problem of “spike sorting” and clustering of neuronal activity
profiles. It has been cited over 180 times.

Beginning around 2005, his group published a series of new neuroinformatics
tools designed to characterise and understand population-scale activity using
the large-scale multielectrode recording methods being developed. These papers
provided the backbone for a new analytic approach that is now being employed
and extended by systems neuroscience laboratories worldwide. A central
component of the current proposal is to disseminate this approach (and others)
already available within Bonsai, easing its adoption by a wider group of
laboratories that lack in-house informatics expertise. Key foundational papers
include Yu et al., 2006, Yu et al., 2009 and Macke et al., 2011.
%
This remains an active thread of his research, with key recent papers including
Duncker et al., 2019, Rutten et al., 2020 and Soulat et al., 2021.
%
Prof.~Sahani has also developed a number of other neuroinformatics tools
relevant to Bonsai.ML including decoding methods (Yu et al., 2010, Santhanam et
al., 2009, Yu et al., 2000) and cell segmentation methods for optical imaging
(Pachitariu et al., 2013, Bohner et al., 2016). Thus, his expertise well be
valuable for evaluation and guidance.

\paragraph{Dr.~Gon\c{c}alo Lopes} (project co-lead) is the creator of Bonsai and
has sample experience in software development, having transitioned from being
the single author of a software used only by himself to run his PhD
experiments, to managing a software development company serving thousands of
users around the world and collaborating with leading universities and research
centers around the world \ldots

\paragraph{Dr.~Joaquín Rapela} (Research Software Engineer) specializes in
signal processing and machine learning, with applications to understanding
brain function. He earned his PhD in Signal Processing from the University of
Southern California, where he developed statistical models to characterize
responses of visual cell to natural images (Rapela et al., 2006, Rapela et al.,
2010). He completed postdoctoral research first using statistical methods to
model cognition in humans at University of California San Diego (Rapela et al.,
2018), and later using machine learning to characterize aberrant neural
activity in epileptic patients at Brown University (Rapela et al., 2019).

Dr.~Rapela has extensive software development expertise, holding a Master’s
degree in Computer Science and industry experience at IBM Argentina and the IBM
Almaden Research Center (San Jose, CA).

Since joining the Gatsby Computational Neuroscience Unit in 2019 as a Research
Engineer Fellow, he has focused on distributing efficient software
implementations of advanced statistical methods created at the Unit, and
supporting experimental neuroscientists in applying advanced computational
tools to analyze neural data.

He is the lead developer of \href{https://github.com/joacorapela/svGPFA}{Sparse
Variational Gaussian Process Factor Analysis (svGPFA)}, and has openly released
several machine learning packages, including linear dynamical systems in
\href{https://github.com/joacorapela/ssm}{Python} and
\href{https://github.com/joacorapela/kalmanFilter}{R},
\href{https://github.com/joacorapela/hiddenMarkovModels}{Hidden Markov Models}
in R, and
\href{https://github.com/joacorapela/bayesianLinearRegression}{Bayesian Linear
Regression} in Python.

Dr.~Rapela played a leading role in securing the BBSRC grant that funded the
creation of Bonsai.ML and has led its development since the project’s
inception. He also played a central role in preparing the current proposal.

\paragraph{Dr.~Nicholas Guilbeault} (research software engineer) has strong
training in biology and systems neuroscience, with a PhD from the department of
cell and systems biology of the University of Toronto.
%
Since the beginning of his PhD, has has been working on the interface between
neuroscience ans software development, especially in Bonsai. He create the
\href{https://ncguilbeault.github.io/BonZeb/}{Bonzeb} package.
%
He joined the Gatsby Unit in 2023 as an RSE for Bonsai.ML and became the lead
software developer of the package.
%
As such he has been exposed to varied machine learning methods, and has
received formal training in advanced mathematical subjects as participants of
the
\href{https://www.ucl.ac.uk/gatsby/study-and-work/gatsby-bridging-programme}{Gatsby
Bridging Program}.

\subsection*{Balance of Skills and Expertise}

Our team has the required expertise, at the leadership and development levels,
in machine learning (MS, JR), software development (GL, JR, NG), neuroscience
(TMF, MS, NG, GL, JR) and experimental control (GL, NG).

Our expertise is complemented by that of world-class project partners in
close-loop neural control (Prof.~Garrett Stanley) and probabilistic programming
(Dr.~Tom Minka), required for the integration activities; and that in
the hippocampus (Prof.~Tim Behrens), high-channel-count electrophysiological
recordings (Dr.~Josh Siegle) and vision and navigation (Prof.~Aman Saleem),
required for the collaborative activities. Please refer to their letter of
support.

\subsection*{Leadership and Management Skills}

\subsection*{Contribution to Developing Good Practice in Communities}

\subsection*{Developing others}

\subsection*{Closing Statement}

