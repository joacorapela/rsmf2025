
\subsection{Relevant Experience}

\paragraph{TMF} is an experimental
neuroscientists and director of the SWC. Research in his laboratory aims to
explain how the brain makes decisions by combining sensory information with
previously learned knowledge. As the behavioural tasks used in his lab require
complex software-control of data acquisition and data analysis pipelines, he
knows first-hand their crucial importance for driving and enabling
neuroscientific research.

He has published 49 peer-reviewed papers, with an h-index of 35 (calculated by
google scholar).  He is a founding member of the International Brain Laboratory
(IBL et al 2019 Neuron). The Bonsai ecosystem is critical to IBL, as it ensures
that experimental control, stimulus presentation and data acquisition can be
identically reproduced across all participating labs in UK, Europe and USA (IBL
et al 2021, eLife)

\paragraph{MS} is a computational neuroscientist and director of the Gatsby
Unit. He has authored over 150 peer-reviewed scientific papers, with an h-index
(computed by Google scholar) of 57. A substantial component of his research
focuses on the development of advanced machine-learning tools for
neuroscience research.

Beginning around 2005, his group published a series of new neuroinformatics
tools designed to characterise and understand population-scale activity using
the large-scale multielectrode recording methods. These papers
provided the backbone for a new analytic approach that is now being employed
and extended by systems neuroscience laboratories worldwide.
%
A central component of the current proposal is to disseminate this approach
(and others) already available within Bonsai, easing its adoption by a wider
group of laboratories that lack in-house informatics expertise.

\paragraph{GL} is the creator of Bonsai and has ample experience in software
engineering, holding a Licentiate degree in Computer Science from NOVA
University Lisbon, and having worked between 2006 and 2010 at the NOVA CENTRIA
Artificial Intelligence laboratory, and at YDreams, where he was leading a team
developing a ground-breaking engine for Augmented Reality.

Transitioning into his neuroscience PhD at the Champalimaud Foundation, he
created the Bonsai visual programming language to run his PhD experiments,
which then led him to managing a software development company serving thousands
of users and collaborating with leading universities and research centres
around the world.

\paragraph{JR} specializes in
signal processing and machine learning, with applications to understanding
brain function (Rapela et al., 2006, Rapela et al., 2010, Rapela et al., 2018
and Rapela et al., 2019).

He has extensive software development expertise, holding a Master’s degree in
Computer Science and industry experience at IBM Argentina and the IBM Almaden
Research Center, US.
%
He joined the Gatsby Computational Neuroscience Unit in 2019 as a Research
Engineer Fellow.
%
He is the lead developer of \href{https://github.com/joacorapela/svGPFA}{Sparse
Variational Gaussian Process Factor Analysis (svGPFA)}, and has openly released
several other machine learning packages including linear dynamical systems in
\href{https://github.com/joacorapela/ssm}{Python} and
\href{https://github.com/joacorapela/kalmanFilter}{R},
\href{https://github.com/joacorapela/hiddenMarkovModels}{Hidden Markov Models}
in R, and
\href{https://github.com/joacorapela/bayesianLinearRegression}{Bayesian Linear
Regression} in Python.

JR played a leading role in securing the BBSRC grant that funded the
creation of Bonsai.ML and has led its development since the project’s
inception. He also played a central role in preparing the current proposal.

\paragraph{NG} is a research software engineer with expertise in real-time
machine learning, neural data analysis, and open-source software development.
He holds a PhD in neuroscience from the University of Toronto, where he
developed the \href{https://ncguilbeault.github.io/BonZeb/}{BonZeb} software
for zebrafish kinematic tracking, closed-loop stimulation, and neural data
analysis.

Since joining the Gatsby Computational Neuroscience Unit in 2023,
Dr.~Guilbeault has served as the core developer of the
\href{https://bonsai-rx.org/machinelearning}{Bonsai.ML} project, integrating
machine learning methods into the Bonsai visual reactive programming language.

\subsection{Balance of Skills and Expertise}

Our team has the required expertise, at the leadership and development levels,
in machine learning (MS, JR, NG), software development (GL, JR, NG), neuroscience
(TMF, MS, GL, JR, NG) and experimental control (GL, NG).

Our expertise is complemented by that of world-class project partners in
close-loop neural control (Prof.~Garrett Stanley), probabilistic programming
(Dr.~Tom Minka), high-channel-count electrophysiological recordings (Dr.~Josh
Siegle) and vision and navigation (Prof.~Aman Saleem).  Please refer to their
letters of support.

\subsection{Leadership and Management Skills}

\paragraph{TMF} is the Director of the Sainsbury Wellcome Centre (SWC) at
University College London. He is responsible for setting the Centre’s strategic
and scientific direction, which currently comprises 12 experimental labs. In
addition, he has line management responsibilities for the Executive Team,
several members of the SWC Faculty, and numerous scientific and administrative
support staff.

\paragraph{MS} has been Director of the Gatsby Computational
Neuroscience Unit since 2017, leading strategy in research and teaching. He has overseen
recruitment to new strategic roles, as well as new members of the Unit faculty.
As Director, he sits in the Executive
Leadership Committee of the Faculty of Life Sciences at UCL.

\paragraph{GL} is the Founder and Director of NeuroGEARS since 2017, where he
has led a diverse team of scientists, engineers and artists in the development
of novel experimental platforms across multiple model organisms. He has also
directly contributed in the organisation and teaching of Bonsai for
neuroscience experimentation worldwide, and was part of the creation of the
Neuronauts educational outreach programme. Currently, NeuroGEARS employs 8
scientists and engineers across the UK, US and Portugal.

\paragraph{JR} has lead a small team of RSEs in the
creation of the Bonsai.ML package since 2023.

\subsection{Contribution to Developing Good Practice in Communities}

Bonsai is an open-source platform built with robust software engineering
practices, including modular design, automated testing, semantic versioning,
and comprehensive documentation. Its development is deeply community-driven,
with contributions from labs and companies worldwide, training workshops,
conferences, and active online forums. Sustained by public/private funding and
the Bonsai Foundation CIC, Bonsai promotes transparency, reproducibility, and
long-term sustainability in neuroscience software.  \subsection{Developing
others}

\paragraph{TMF} has led a laboratory since 2008. He has
successfully supervised 10 MA students, 7 PhD students, and 12 postdoctoral
scholars.

\paragraph{MS} has led a laboratory since 2004, and has supervised a total of 18 PhD students and
17 postdoctoral fellows.

\paragraph{GL} has supervised and trained over 10 scientists and engineers at
NeuroGEARS since 2017, and co-supervised 2 iCASE studentships in collaboration
with UCL faculty.

\paragraph{JR} co-mentored a masters, two undergraduate
students and is advising an RSE in machine learning methods development.
