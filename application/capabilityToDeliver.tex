The first draft of this proposal was prepared by the senoir RSE, JR, with
contributions from the junior RSE, NG. The senior RSE could not be project lead
due to UCL policies.
%
RSEs at the Gatsby Unit and SWC are deeply immerse in a machine-learning and
systems-neuroscience research environment, and their work is focused on
developing machine learning software for systems neuroscience research.
%
Thus, they are well prepared to deliver the proposed activities.

\textbf{JR} has ample \textbf{software engineering} training and experience, with a
masters degree in computer science, work experience at IBM Argentina and the
IBM Almaden Research Center (San Jose, CA), and six years of RSE experience at
the Gatsby Unit.
%
He is the lead developer of the package
\href{https://github.com/joacorapela/svGPFA}{Sparse Variational Gaussian
Process Factor Analysis} and has openly distributed several machine learning
methods, including linear dynamical system models in
\href{https://github.com/joacorapela/ssm}{Python} and
\href{https://github.com/joacorapela/kalmanFilter}{R},
\href{https://github.com/joacorapela/hiddenMarkovModels}{Hidden Markov Models}
in R, and
\href{https://github.com/joacorapela/bayesianLinearRegression}{Bayesian Linear
Regression} in Python.
%
JR is skilled in \textbf{signal processing and ML}, with a PhD in Signal Processing from
the Electrical Engineering Department, University of Southern California,
postdoctoral experience using machine learning to model human
electrophysiological recordings at the Swartz Center for Computational
Neuroscience, University California San Diego (UCSD), and at Brown University, and
machine learning software development practice at the Gatsby Unit.
%
He has applied his quantitative skills to \textbf{understand the function of
the brain}. He modeled the visual system during his PhD, he studied attention in humans
during his postdoctoral position at UCSD, he investigated epilepsy in humans during his postdoc
at Brown University, and contributed to investigations on diverse aspects of brain function in rodents as a RSE
at the Gatsby Unit and SWC.
%
JR has successful \textbf{funding acquisition experience}. During his first
postdoctoral position at UCSD, he was awarded a seed
grant as principal investigator (\$30.000) to model neural ensembles with population dynamical models
using human ECoG recordings during speech production; and in 2022 he prepared
the first draft of the Bonsai.ML proposal to BBSRC (\pounds 800.000).
%
Since 2023 he has been \textbf{leading the development of this project}.


NG has strong training in biology and systems neuroscience, has extensive
software development experience and he is now skilled in mathematics.

The RSEs assembled a unique multidisciplinary team to supervise and develop the
proposed activities, combining critical expertise in software engineering, ML,
and both experimental and computational neuroscience.
%
The project lead, TMF, directs the SWC, a world-class experimental neuroscience
research center, while the project co-lead, MS, directs the Gatsby Unit, a meca
in machine learning and computational neuroscience. Both research centers share
a common building and work in close collaboration.
%
The external project co-lead, GL, is the creator of Bonsai and has sample
experience in software development, having transitioned from being the single
author of a software used only by himself to run his PhD experiments, to
managing a software development company serving thousands of users around the
world and collaborating with leading universities and research centers around
the world.



\subsection*{Highly trained RSEs in }

Acquiring domain expertise is essential for the work of RSEs. Ideally, RSEs
are well trained in the domain area of their collaborating scientists, and
attend their research talks. However, in standard university RSE groups, where
RSEs work with scientists in many domain areas, acquiring this domain expertise
is very challenging.

Fortunately, at the SWC and Gatsby Unit, RSEs are deeply integrated into the
machine learning and neuroscience research of their collaborating scientists,
they attend their research talks, and interact closely with them.  Hence, our
RSEs are highly trained in machine learning and neuroscience, skills that are
central for the successful delivery of this proposal.
