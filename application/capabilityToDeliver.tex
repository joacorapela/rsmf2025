
\subsection*{Relevant Experience}

\paragraph{TMF} is an experimental
neuroscientists and director of the SWC. Research in his laboratory aims to
explain how the brain makes decisions by combining sensory information with
previously learned knowledge.  As the behavioural tasks used in his lab require
complex software-control of data acquisition and data analysis pipelines, he
knows first-hand their crucial importance for driving and enabling
neuroscientific research.

He has published 49 peer-reviewed papers, with an h-index of 35 (calculated by
google scholar).  He is a founding member of the International Brain Laboratory
(IBL et al 2019 Neuron). The Bonsai ecosystem is critical to IBL, as it ensures
that experimental control, stimulus presentation and data acquisition can be
identically reproduced across all participating labs in UK, Europe and USA (IBL
et al 2021, eLife)

\paragraph{MS} is a computational neuroscientists and director of the Gatsby
Unit. He has authored over 100 peer-reviewed scientific papers, with an h-index
(computed by Google scholar) of 47. A substantial component of his research
focuses on the development of advance machine-learning based tools for
neuroscience research.

Beginning around 2005, his group published a series of new neuroinformatics
tools designed to characterise and understand population-scale activity using
the large-scale multielectrode recording methods being developed. These papers
provided the backbone for a new analytic approach that is now being employed
and extended by systems neuroscience laboratories worldwide.
%
Key papers include Yu et al., 2006, Yu et al., 2009 and Macke et al., 2011,
Duncker et al., 2019, Rutten et al., 2020 and Soulat et al., 2021.
%
A central component of the current proposal is to disseminate this approach
(and others) already available within Bonsai, easing its adoption by a wider
group of laboratories that lack in-house informatics expertise.

\paragraph{GL} is the creator of Bonsai and
has sample experience in software development, having transitioned from being
the single author of a software used only by himself to run his PhD
experiments, to managing a software development company serving thousands of
users around the world and collaborating with leading universities and research
centers around the world \ldots

\paragraph{JR} specializes in
signal processing and machine learning, with applications to understanding
brain function (Rapela et al., 2006, Rapela et al., 2010, Rapela et al., 2018
and Rapela et al., 2019).

He has extensive software development expertise, holding a Master’s degree in
Computer Science and industry experience at IBM Argentina and the IBM Almaden
Research Center, US.
%
He joined the Gatsby Computational Neuroscience Unit in 2019 as a Research
Engineer Fellow.
%
He is the lead developer of \href{https://github.com/joacorapela/svGPFA}{Sparse
Variational Gaussian Process Factor Analysis (svGPFA)}, and has openly released
several other machine learning packages including linear dynamical systems in
\href{https://github.com/joacorapela/ssm}{Python} and
\href{https://github.com/joacorapela/kalmanFilter}{R},
\href{https://github.com/joacorapela/hiddenMarkovModels}{Hidden Markov Models}
in R, and
\href{https://github.com/joacorapela/bayesianLinearRegression}{Bayesian Linear
Regression} in Python.

Dr.~Rapela played a leading role in securing the BBSRC grant that funded the
creation of Bonsai.ML and has led its development since the project’s
inception. He also played a central role in preparing the current proposal.

\paragraph{NG} has strong
training in biology and systems neuroscience, with a PhD from the department of
cell and systems biology of the University of Toronto.
%
Since the beginning of his PhD, has has been working on the interface between
neuroscience ans software development, especially in Bonsai. He created the
\href{https://ncguilbeault.github.io/BonZeb/}{Bonzeb} package.
%
He joined the Gatsby Unit in 2023 as an RSE for Bonsai.ML and became the lead
software developer of the package.
%
As such he has been exposed to varied machine learning methods, and has
received formal training in advanced mathematical subjects as participants of
the
\href{https://www.ucl.ac.uk/gatsby/study-and-work/gatsby-bridging-programme}{Gatsby
Bridging Program}.

\subsection*{Balance of Skills and Expertise}

Our team has the required expertise, at the leadership and development levels,
in machine learning (MS, JR), software development (GL, JR, NG), neuroscience
(TMF, MS, NG, GL, JR) and experimental control (GL, NG).

Our expertise is complemented by that of world-class project partners in
close-loop neural control (Prof.~Garrett Stanley) and probabilistic programming
(Dr.~Tom Minka), required for the integration activities; and that in the
hippocampus (Prof.~Tim Behrens), high-channel-count electrophysiological
recordings (Dr.~Josh Siegle) and vision and navigation (Prof.~Aman Saleem),
required for the collaborative activities. Please refer to their letters of
support.

\subsection*{Leadership and Management Skills}

\paragraph{TMF} is the Director of the Sainsbury Wellcome Centre (SWC) at
University College London. He is responsible for setting the Centre’s strategic
and scientific direction, which currently comprises 12 experimental labs. In
addition, he has line management responsibilities for the Executive Team,
several members of the SWC Faculty, and numerous scientific and administrative
support staff.

\paragraph{MS} has been Director of the Gatsby Computational
Neuroscience Unit since 2017, leading strategy in research and teaching. He has
created two new roles to support and develop new strategy, and has recruited
two new members of faculty who.  As Director, he sits in the Executive
Leadership Committee of the Faculty of Life Sciences at UCL.

\paragraph{GL} founded NeuroGEARS, the main contributor to
Bonsai, in 2017, where he leads a team of scientists, engineers and artists.
Currently NeuroGEARS employs xx scientists/engineers.

\paragraph{JR} has lead a small team of RSEs in the
creation of the Bonsai.ML package since 2023.

\subsection*{Contribution to Developing Good Practice in Communities}

At Bonsai courses student learn about version control, testing, modular
programming and, importantly, good graphical programming patterns in Bonsai.
%
At \href{https://github.com/orgs/bonsai-rx/discussions}{Bonsai discussions}
users present problems, discuss their solutions and assimilate best
Bonsai practices.

JR has contributed to software carpentry courses.


\subsection*{Developing others}

\paragraph{TMF} has led a laboratory since 2008. He has
successfully supervised 10 MA students, 7 PhD students, and 12 postdoctoral
scholars.

\paragraph{MS} has supervised a total of 14 PhD students and
14 postdoctoral fellows.

\paragraph{GL} has trained xxx engineers at NeuroGEARS.

\paragraph{JR} co-mentored a masters, two undergraduate
students and is advising NG in machine learning methods development.
