We are committed to ensuring that Bonsai.ML is accessible, inclusive, and 
beneficial to the widest possible community of neuroscience researchers.

The Bonsai.ML team itself is diverse, with leadership from Croatia, India, 
and Portugal, and research software engineers from Argentina and Canada. 
This diversity shapes our perspective and makes us strong advocates of EDIA. 
Bonsai already has a global user base spanning all five continents. It 
empowers researchers in disadvantaged communities, notably in Eastern Europe 
and South America. A recent example is the Transatlantic Behavioral 
Neuroscience School (Argentina, August 2025), where Bonsai was used to enable 
cutting-edge training opportunities across borders.

At its core, Bonsai embodies inclusivity: it allows non-programmers to design 
and run sophisticated experiments. This lowers barriers for researchers from 
underrepresented groups, smaller institutions, or disciplines outside computer 
science---broadening participation in methods that would otherwise remain the 
preserve of elite or technically specialized groups. With Bonsai.ML, we extend 
this philosophy by providing non-programmers with state-of-the-art machine 
learning tools.

The activities supported by this grant will further embed EDIA principles. 
Comprehensive documentation and training will lower entry barriers through 
step-by-step tutorials, plain-language explanations, captioned video materials, 
and diverse experimental examples. All materials will be openly available and 
designed with accessibility in mind (e.g.\ screen-reader compatibility, 
colourblind-friendly figures). Our documentation approach will closely follow 
the \texttt{scikit-learn} project, which is internationally recognized for 
embracing EDIA principles through clarity, consistency, and accessible 
contribution pathways.

Governance and mentorship will also reflect EDIA priorities. The Bonsai.ML 
steering committee will include members from multiple disciplines and
institutions, 
and we will involve early-career researchers through mentoring and clear 
contribution pathways, ensuring that underrepresented groups have a voice in 
shaping the long-term development of the platform.

Finally, community events such as the 2026 Bonsai Developers Conference will 
adopt a code of conduct, select a diverse range of speakers across career 
stages, genders, and regions, and provide hybrid participation options to
reduce 
financial and travel barriers.

