\subsubsection*{Timeliness}

The last few years have demonstrated the transformative potential of machine
learning in biology. For example, the 2024 Nobel Prize in Chemistry was awarded
to Demis Hassabis, a former Gatsby Unit PhD student, for his pioneering work on
applying ML to protein structure prediction. Similar opportunities now exist
for integrating ML into experimental control in neuroscience and biology.
Capturing these opportunities quickly is essential to ensure that the UK
remains at the forefront of this field.

Support from this award is also critical for retaining a key Bonsai.ML
developer, ensuring continuity in the project and avoiding disruption to a
highly skilled and experienced contributor.

\subsubsection*{Other sources of support}

We have considered other funding mechanisms. A BBSRC Standard Research Grant is
not appropriate at this stage, as Bonsai.ML is not yet embedded in a major
biological research programme.
%
Likewise, a BBSRC Follow-on Fund is not a fit, as we cannot currently point to
a single application of Bonsai.ML with immediate transformative economic or
societal impact.
%
The RSMF is uniquely well-suited to support the software maintenance and
integration work we propose.

\subsubsection*{Sustainability}

Bonsai.ML will be sustained beyond the RSMF funding period through a
combination of community, institutional, and commercial support.
%
NeuroGEARS, which already underpins the wider Bonsai ecosystem, invests a fixed
proportion of its service income into Bonsai maintenance and will extend this
support to Bonsai.ML.

Our academic partners are also strongly invested.
%
The Sainsbury Wellcome Centre
(SWC) relies on Bonsai for experimental control and Bonsai.ML for advanced
control in some of its experiments, and contributes financially to its
development.
%
The Gatsby Unit will continue to provide machine learning
expertise to guide Bonsai.ML’s growth.
%
Prof. Stanley’s lab will contribute its expertise in the control of
physiological signals to extend Bonsai.ML’s capabilities.

Finally, this project initiates a long-term collaboration between Microsoft
Research Cambridge, the Gatsby Unit, the SWC, and NeuroGEARS (see Dr. Minka’s
support letter). Integrating Infer.NET into Bonsai.ML will broaden Infer.NET’s
reach into neuroscience and strengthen its user base, while Bonsai.ML users
will benefit from direct access to Infer.NET expertise.
