\subsubsection*{Timeliness}

The last few years have demonstrated the transformative potential of ML in
biology. For example, the 2024 Nobel Prize in Chemistry was awarded to Demis
Hassabis, a former Gatsby Unit PhD student, for his pioneering work on applying
ML to protein structure prediction. New opportunities now exist for
integrating ML into experimental control in neuroscience and biology.
Capturing these opportunities quickly is essential to ensure that the UK
remains at the forefront of this field.

\subsubsection*{Other sources of support}

We have considered other funding mechanisms. A BBSRC Standard Research Grant is
not appropriate at this stage, as Bonsai.ML is not yet embedded in a major
biological research programme.
%
Likewise, a BBSRC Follow-on Fund is not a fit, as we cannot currently point to
a single application of Bonsai.ML with immediate transformative economic or
societal impact.

\subsubsection*{Sustainability}

Bonsai.ML will be sustained beyond the RSMF funding period through a
combination of community, institutional, and commercial support.
%
NeuroGEARS, which already underpins the wider Bonsai ecosystem, invests a fixed
proportion of its service income into Bonsai maintenance and will extend this
support to Bonsai.ML.

Our academic partners are also strongly invested.
%
The Sainsbury Wellcome Centre
(SWC) relies on Bonsai for experimental control and Bonsai.ML for advanced
control in some of its experiments, and contributes financially to its
development.
%
The Gatsby Unit will continue to provide machine learning
expertise to guide Bonsai.ML’s growth.
%
Prof. Stanley’s lab will contribute its expertise in the control of
physiological signals to extend Bonsai.ML’s capabilities.

Finally, our collaboration with Microsoft Research Cambridge ensures that 
Infer.NET integration brings sustained industrial expertise to Bonsai.ML, 
strengthening its long-term sustainability beyond the funding period.

