
\subsection{Approach}

\subsubsection{Aim 1: Documentation}

The current Bonsai.ML documentation includes installation guides, API
references, and examples, but it lacks essential statistical explanations,
clear connections between methods, and a well-structured organisation. Users
have reported redundancy, gaps in clarity, and difficulty finding information,
creating friction for newcomers. In addition, documentation for machine
learning method developers is scattered across disparate sources rather than
being centralised. Inspired by the model of
\href{https://scikit-learn.org/}{scikit-learn}, which combines conceptual
background with tutorials and API references, we aim to restructure and expand
Bonsai.ML documentation into a comprehensive, accessible resource that serves
both experimental neuroscientists and method developers.

\paragraph{Tasks:}  

\begin{description}
    \item[doc\_structure:] Restructure documentation in the style of \texttt{scikit-learn} (4 weeks).  
    \item[doc\_examples:] Add at least 3 neuroscience-focused examples (3 weeks).  
    \item[doc\_trouble:] Create a troubleshooting section with common issues (2 weeks).  
    \item[doc\_videos:] Produce 3 tutorial videos, hosted on YouTube and embedded in the docs (3 weeks).  
    \item[doc\_cases:] Develop 2 case studies on intelligent experimental control in Bonsai (4 weeks).  
\end{description}  

\paragraph{Milestones and Indicators:}  

\begin{description}
    \item[doc\_m1/i1:] Documentation restructured and published on the Bonsai.ML site.  
    \item[doc\_m2/i2:] Three new neuroscience examples added and published.  
    \item[doc\_m3/i3:] Troubleshooting section online with at least 10 issues addressed.  
    \item[doc\_m4/i4:] Three tutorial videos produced, uploaded, and linked in the docs.  
    \item[doc\_m5/i5:] Two case studies completed and published in a new Case Studies section.  
\end{description}

\paragraph{Responsible team members:} NG

\noindent\rule{\textwidth}{1pt}
\subsubsection{Aim 2: Training}
\paragraph{Summary:} Organise a Bonsai course with a dedicated Bonsai.ML module.

\paragraph{Background:} Since 2017, NeuroGEARS Ltd has organised at least two
Bonsai courses per year at different universities around the world, and some of
them can be \href{https://bonsai-rx.org/learn/}{viewed online}.
%
We will deliver a Bonsai course, with a new Bonsai.ML module. It will be
targeted to intermediate Bonsai users, and will take place at the lecture
theatre of the Sainsbury Wellcome Centre, which is free for us. It will host a
class size of around 20 students. The structure of the course will be similar
to that of previous ones (e.g.,
\href{https://neurogears.org/st-andrews-2024/}{2024 Bonsai Course at
St.~Andrews University}), with the addition of a Bonsai.ML module.

\paragraph{Tasks:}  

\begin{description}
    \item[course\_prep] Prepare course syllabus and teaching materials; secure venue; announce the course; recruit instructors, teaching assistants, and students. (3 weeks)  
    \item[course\_del] Deliver the course. (1 week)  
\end{description}  

\paragraph{Milestones and Indicators:}  

\begin{description}
    \item[course\_m1] Course publicly announced with detailed syllabus.  
    \item[course\_i1] Announcement disseminated via multiple channels (mailing lists, X, Mastodon, Linked In) with support from SWC and Gatsby Unit communication teams.  
    \item[course\_m2] Course delivered.  
    \item[course\_i2] Course recordings, slides, worksheets, and solutions made freely available on GitHub.  
\end{description}

\paragraph{Responsible team members:} GL

\noindent\rule{\textwidth}{1pt}
\subsubsection{Aim 3: Dissemination}

\paragraph{Summary:} Publish first Bonsai.ML paper

\paragraph{Background:} Specially in neuroscience, adoption of open-source
software greatly increase when they are supported by scientific papers. In our
own experiences, we have seen this happen many times, with the foundational
Bonsai paper \citep{lopesEtAl15}, with the BonVision paper \citep{lopesEtAl21},
and with the BonZeb paper \citep{guilbeaultEtAl21}.
%
To increase adoption of Bonsai.ML, we will prepare and submit for publication
the first Bonsai.ML paper.

\paragraph{Tasks:}\mbox{}\\

\begin{description}

    \item[paper\_write:] Write first Bonsai.ML paper (4 weeks).

\end{description}

\paragraph{Milestones and Indicators:}\mbox{}\\

\begin{description}

    \item[paper\_m1]: First Bonsai.ML paper written.
    \item[paper\_i1]: First Bonsai.ML paper available in
        \href{https://www.biorxiv.org/}{BioRxiv}.

\end{description}

\noindent\rule{\textwidth}{1pt}
\subsubsection{Aim 4: Maintainability}

\paragraph{Summary:} Most probabilistic models in Bonsai.ML are currently
implemented in Python, which helps attract Python developers but limits
performance in real-time applications. Implementations in C\# and especially in
a probabilistic programming language (PPL) would be faster, simpler, and more
homogeneous, since PPLs abstract away the complexity of inference algorithms.
Integrating a PPL would make adding new models far easier and greatly improve
maintainability.  

Fortunately, C\# has a mature PPL:
\href{https://dotnet.github.io/infer/}{Infer.NET}, developed at Microsoft
Research Cambridge, used in hundreds of papers, and open-sourced in 2018.
Infer.NET delivers scalable, deterministic approximate inference, enabling
powerful models to be implemented in only a few lines of code. Incorporating it
into Bonsai.ML will accelerate inference, standardise implementations, and make
it much easier for neuroscientists to extend the ecosystem.  

Our collaborator Dr.~Tom Minka, inventor of Expectation Propagation and lead
developer of Infer.NET, will support this effort.  

\paragraph{Tasks:}  

\begin{description}
    \item[in\_learn:] The responsible team member, already skilled in probabilistic programming, will dedicate two weeks to learning Infer.NET (2 weeks).  

    \item[in\_OBLR:] Re-implement the Online Bayesian Linear Regression (currently in C\#) using Infer.NET. Develop test cases to validate equivalence with the existing implementation (2 weeks).  

    \item[in\_LDS:] Re-implement the Linear Dynamical System model (currently in Python) in Infer.NET, with test cases to ensure equivalence (2 weeks).  

    \item[in\_HMM:] Re-implement the Hidden Markov Model (currently in Python) in Infer.NET, with test cases to ensure equivalence (2 weeks).  

    \item[in\_PPdecoder:] Re-implement the Point Process Decoder (currently in Python) in Infer.NET, with test cases to ensure equivalence (2 weeks).  

    \item[in\_docs:] Add detailed documentation for each model integrated above, describing implementation in Infer.NET (3 weeks).  

\end{description}  

\paragraph{Milestones and Indicators:}  

\begin{description}
    \item[in\_m1/i1:] OBLR implemented in Bonsai.ML.Infer.NET and published on \texttt{nuget.org}.  
    \item[in\_m2/i2:] LDS implemented in Bonsai.ML.Infer.NET and published on \texttt{nuget.org}.  
    \item[in\_m3/i3:] HMM implemented in Bonsai.ML.Infer.NET and published on \texttt{nuget.org}.  
    \item[in\_m4/i4:] Point Process Decoder implemented in Bonsai.ML.Infer.NET and published on \texttt{nuget.org}.  
    \item[in\_m5/i5:] Comprehensive documentation added for all models and published with the corresponding \texttt{nuget} packages.  
    \item[in\_m6/i6:] All Bonsai.ML.Infer.NET packages released and available on \texttt{nuget.org}.  
\end{description}

\paragraph{Responsible team member:} JR.

\noindent\rule{\textwidth}{1pt}
\subsubsection{Aim 5: Community reach}

\paragraph{Summary:} Attract to Bonsai scientists interested in close-loop
control of neural activity by integrating functionality from
the \href{https://cloctools.github.io/}{CLOCTools} software into Bonsai.ML.

We will expand Bonsai.ML to support closed-loop neural control by integrating
functionality from \href{https://cloctools.github.io/}{CLOCTools}
, developed by Prof.~Garrett Stanley’s lab (Georgia Tech/Emory). While Bonsai
already excels in real-time behavioural control, this integration will make it
the first sustainable, general-purpose framework supporting both behavioural
and neural closed-loop experimentation. Existing closed-loop tools are often ad
hoc, difficult to install, and poorly integrated; Bonsai will provide
accessible, open-source, and maintainable solutions, lowering barriers to
adoption and enabling real-time read–write interaction with neural activity.
Concretely, we will create a Bonsai package that interfaces with CLOCTools’
\href{https://github.com/CLOCTools/lds-ctrl-est}{lds-ctrl-est}
 repository, with Prof.~Stanley’s team maintaining the core library and our
 team providing Bonsai integration.

\paragraph{Tasks:}  

\begin{description}
    \item[ldsCE\_eval:] Review and evaluate the functionality of \href{https://github.com/CLOCTools/lds-ctrl-est}{lds-ctrl-est}. (2 weeks)  
    \item[ldsCE\_wrapper:] Create C++/CLI wrappers to access \texttt{lds-ctrl-est} from C\#. (3 weeks)  
    \item[ldsCE\_integration:] Build the \texttt{Bonsai.ML.LDS-CTR-EST} package providing \texttt{lds-ctrl-est} functionality within Bonsai. (3 weeks)  
    \item[ldsCE\_testing:] Evaluate with synthetic data to ensure real-time performance (2 weeks), and assist Prof.~Stanley’s lab in validation with experimental data (4 weeks).  
    \item[ldsCE\_docs:] Add documentation in line with other Bonsai.ML packages. (3 weeks)  
    \item[ldsCE\_release:] Release the package. (1 week)  
\end{description}  

\paragraph{Milestones and Indicators:}  

\begin{description}
    \item[ldsCE\_m1/i1:] Core functionality accessible from C\#, with examples from \texttt{lds-ctrl-est} replicated in C\# and shared on GitHub.  
    \item[ldsCE\_m2/i2:] Package integrated into Bonsai and available via the package manager.  
    \item[ldsCE\_m3/i3:] Package validated with synthetic data; latency figures documented.  
    \item[ldsCE\_m4/i4:] Experimental validation in collaboration with Prof.~Stanley’s lab, with results (e.g. steady-state firing rate control as in \citet{bolusEtAl21}) documented on GitHub.  
    \item[ldsCE\_m5/i5:] API documentation, examples, and tutorials completed.  
    \item[ldsCE\_m6/i6:] Package released and available on \texttt{nuget.org}.  
\end{description}

\paragraph{Responsible team members:} JR

\noindent\rule{\textwidth}{1pt}
\subsubsection{Aim 6: Community building}
\paragraph{Summary:} Organise the 2026 Bonsai developers conference

\paragraph{Background:} A Bonsai developer conference is a week-long
meetings aimed at bringing together neuroscience researchers, computational
scientists, and software engineers who are interested in developing and using
the Bonsai visual reactive programming language.

The first edition of this conference took place at the Sainsbury Wellcome
Centre on December 2024. It was attended by 30 scientists and engineers from
around the world, including representatives from the Allen Institute for Neural
Dynamics, Janelia Research Campus, Massachusetts Institute of Technology, and
University College London, to mention a few.
%
The website of the first edition of this conference  can be found
\href{https://conference.bonsai-rx.org/2024/}{here}.  The proceedings
repository of the first conference is still under construction, but can be
accessed from
\href{https://github.com/joacorapela/bonsaiConference2024Proceedings}{here},
and the summary of its Bonsai.ML session can be read
\href{https://github.com/joacorapela/bonsaiConference2024Proceedings/blob/master/sessions/machineLearning/README.md}{here}.

In this first edition of the conference we voted to hold the second edition in
2026. We propose to organise and deliver it as part of this project.

\paragraph{Tasks:}  

\begin{description}
    \item[conf\_prep:] Prepare program, invite speakers, arrange venue and logistics, announce event, and select participants (3 weeks).  
    \item[conf\_del:] Deliver the conference (1 week).  
    \item[conf\_proc:] Prepare proceedings (2 weeks).  
\end{description}  

\paragraph{Milestones and Indicators:}  

\begin{description}
    \item[conf\_m1/i1:] Event announced via mailing lists and social media (X, Mastodon), with preliminary program and speaker list.  
    \item[conf\_m2/i2:] Event delivered; raw video recordings shared with the Research Software Management Fund.  
    \item[conf\_m3/i3:] Proceedings released in a GitHub repository, including attendance list, recordings, narrative summaries, slides, and exercises with solutions (cf. \href{https://github.com/joacorapela/bonsaiConference2024Proceedings}{2024 conference proceedings}).  
\end{description}

\noindent\rule{\textwidth}{1pt}
\subsubsection{Aim 7: Governance}

\paragraph{Background:} We will assemble a Bonsai.ML steering committee that
will be responsible for approving project milestones and advise us on building
a long-term development roadmap for Bonsai.ML.  Responding to guidance and
feedback from the steering committee ensures Bonsai.ML addressed pressing
neuroscience needs on an international scale.

Several renowned experimental and computational neuroscientists around the
world are heavily invested in Bonsai, and are very interested in adding ML
functionality to their Bonsai workflows. We have already invited a few of them
to join the Bonsai.ML steering committee: Prof.~Garrett Stanley, Prof.~Aman
Saleem and Dr.~Josh Siegle. Please refer to their letters of support.

We will meet with the steering committee four times in 2026. These meetings
will happen at the end of January, April, August and November, respectively.
%
Before each meeting, we will prepare a progress report and send it to the
committee members two weeks before the meeting. After each meeting we will
write a follow-up report with meetings minutes and the revised project roadmap.

\paragraph{Tasks:}

\begin{description}

    \item[sc\_assembly:] Complete the assembly of the steering committee.

    \item[sc\_meet\{i\}:] Meet with the steering committee for this ith
        time, with $i\in\{1,2,3,4\}$.

\end{description}

\paragraph{Milestones and Indicators:}\mbox{}\\

\begin{description}

    \item[sc\_m0:] Steering committee assembled.

    \item[sc\_i0:] Documented published in the project GitHub repository with
        names and affiliations of the steering committee members.

    \item[sc\_m\{i\}:] ith steering committee meeting held.

    \item[sc\_i\{i\}] Progress and follow-up reports for the ith meeting.

\end{description}

\noindent\rule{\textwidth}{1pt}
\subsection{Management}

We will conduct management activities at different frequencies:

\begin{description}

    \item[Quarterly:] we will convene the steering committee. Two weeks in
        advance of each meeting, we will circulate a progress report
        summarising achieved milestones, proposed future activities, and the
        most recent version of the Bonsai.ML long-term roadmap.
        %
        The steering committee will review the report, endorse completed
        milestones, and provide feedback or suggestions on planned activities.
        %
        Based on project progress and committee input, we will revise the
        long-term roadmap, ensuring that Bonsai.ML development remains aligned
        with community needs and strategic goals.
        %
        Minutes and key decisions from these meetings will be documented and,
        shared with the Research Software Maintenance Fund.

    \item[Monthly:] we will hold meetings between the project lead, the
        project co-lead, the RSE, and the external project co-lead, to evaluate
        the project progress.

    \item[Weekly:] as has been our practice since the start of the Bonsai.ML
        project, the RSEs will meet with the external project co-lead to
        discuss issues that appeared during the week, review activities for the
        following week, and adjust project directions.

\end{description}

Meetings with collaborators will be arranged as needed.
%
At the SWC, GCNU and NG we are experimental and computational neuroscientists
with successful collaborative experience, and we have no doubt that the
proposed collaborations will be of the same kind,
%
specially since we have successfully interacted in the past with most of the
propose collaborators.
%
Please refer to their letters of support.
