
\begin{figure}
    \centering
    \includegraphics[width=6in]{activitiesGraphs/activities_larger.png}
    \caption{Proposed activities. See text for details.}
\end{figure}

\subsection*{Documentation}

Good documentation is essential for software uptake.
%
Bonsai.ML already contains substantial
\href{https://bonsai-rx.org/machinelearning/index.html}{documentation} for all
distributed packages. This documentation includes Bonsai working examples that users
can copy and paste them in their Bonsai local runtime and execute them. We will
extend this documentation by:

\begin{description}

    \item[providing more examples] on the application of the
        already integrated ML methods to new types of behavioural and
        neural data.

    \item[including video tutorials] on the use of the Bonsai.ML
        package.

    \item[adding documentation for methods developers] as the current one is
        targeted to Bonsai.ML users.

    \item[creating case studies] for intelligent experimental control in Bonsai,
        similar to the case studies for neural data analysis in
        Python\footnote[4]{\url{https://mark-kramer.github.io/Case-Studies-Python/intro.html}},
        or those for model based machine
        learning\footnote[5]{\url{https://mbmlbook.com/index.html}}, by our Microsoft
        collaborators.
        The aim of these case studies is to provide ML training to
        Bonsai users.

    \item[publishing a first Bonsai.ML paper] describing its functionality, as
        companion papers substantially increases the adoption of software
        packages \citep{lopesEtAl15,guilbeaultEtAl21}.

\end{description}

\subsection*{Training}

Since 2017, NeuroGEARS Ltd (the non-profit organisation that is the main
contributor to the development of Bonsai) has organised at least two Bonsai
course per year at different universities, and some of them can be viewed
online\footnote[5]{\url{https://bonsai-rx.org/learn/}}. We will organise a new
Bonsai course with a module on Bonsai.ML.

\subsection*{Community}

Adding state-of-the-art ML methods and comprehensive documentation is critical.
However, from our previous experience with Bonsai, adoption will be low if we
do not demonstrate the impact of Bonsai.ML methods for solving important
neuroscience problems.
%
For this, we need to work closely with experimental neuroscientists to help
them solve relevant problems in intelligent experimental control.

We will collaborate with experimental neuroscientists at the Sainsbury Wellcome
Centre (SWC; e.g., Prof.~Athena Akrami), at the Institute for Behavioral
Neuroscience (IBN) of University College London (Prof.~Aman Saleem), and at the
Allen Institute for Neural Dynamics (AIND; Dr.~Josh Siegle), on the integration
of machine learning functionality that we have already developed into their
experiments.
%
All these scientists are our close collaborators. Please refer to their letters
of support.

\subsection*{Integrations}

\subsubsection*{Probabilistic programming software: Infer.NET}

Most of the probabilistic models currently integrated into Bonsai are
implemented in Python. These serve as excellent demonstrations of how Python
applications can connect to the Bonsai ecosystem, and they are central to our
aim of attracting Python developers to contribute to Bonsai.ML.

However, Python implementations are substantially slower than equivalent C\#
code. For demanding real-time applications, C\# implementations are preferable,
especially when expressed in a probabilistic programming language (PPL).

The existing Python and C\# implementations of probabilistic models (e.g.,
linear dynamical systems, hidden Markov models, Bayesian linear regression —
see
\href{https://bonsai-rx.org/machinelearning/examples/README.html}{examples})
are relatively complex and heterogeneous, making them harder to maintain. In
contrast, implementations in a C\# PPL would be faster, simpler, and more
homogeneous, reducing errors and improving maintainability.

Fortunately, C\# has an excellent PPL:
\href{https://dotnet.github.io/infer/}{Infer.NET}, developed at Microsoft
Research Cambridge since 2004, used in
\href{https://dotnet.github.io/infer/papers.html}{hundreds of papers}, and
\href{https://www.microsoft.com/en-us/research/blog/the-microsoft-infer-net-machine-learning-framework-goes-open-source/}{open-sourced
in 2018}.  Infer.NET uses deterministic approximate inference, enabling fast
and scalable solutions. For example, it has powered systems that extract
knowledge from billions of web pages (petabyte-scale
data\footnote[7]{\url{https://www.microsoft.com/en-us/research/blog/the-microsoft-infer-net-machine-learning-framework-goes-open-source/}})
— the kind of scalability critical for real-time inference in Bonsai.

Message passing is a natural connection point between Bonsai and Infer.NET.
Infer.NET uses message passing for approximate inference, while Bonsai relies
on message passing for reactive computations. Exposing Infer.NET’s message
passing algorithms as Bonsai nodes will allow users to seamlessly combine
efficient inference with reactive experimental control.

We will re-implement in Infer.NET all the probabilistic models previously
integrated into Bonsai using Python or C\#. This integration will accelerate
inference, simplify and standardize inference programs, and empower Bonsai
users to create new probabilistic models and inference algorithms with just a
few lines of code. By making powerful yet easy-to-use inference methods
accessible to experimental neuroscientists, this project has the potential to
profoundly advance scientific discovery.

\subsubsection*{Closed-loop optogenetic control: CLOC}

Closed-loop neural control represents a transformative advance in neuroscience:  
rather than delivering stimulation at fixed, open-loop schedules, it enables  
precisely timed interventions based on ongoing brain and behavioral activity.  
This paradigm allows researchers to move beyond observing correlations to  
directly testing causal mechanisms of neural dynamics, plasticity, and
behavior.  
Critically, closed-loop stimulation has already proved transformative in the  
clinic, for example in deep brain stimulation for Parkinson’s disease and  
epilepsy, while its extension to other disorders (e.g. depression, obsessive  
compulsive disorder) remains an active area of research. Despite this promise,  
widespread adoption has been limited by the lack of accessible,
well-engineered,  
and sustainable software frameworks for real-time experimental control.

Prof.~Garrett Stanley (Georgia Tech, US) is a pioneer in closed-loop  
neuroscience, having developed groundbreaking methods that combine real-time  
neural control with systems neuroscience. His laboratory recently contacted us  
to explore integrating their existing CLOC tools, originally implemented in  
RTXI/C++, into Bonsai.ML\footnote[6]{\url{https://cloctools.github.io/}}.  
This represents a unique opportunity: Bonsai already excels at real-time  
closed-loop control in behavioral experiments, and extending it to include  
state-of-the-art closed-loop neural control will position the platform as the  
first sustainable, general-purpose framework for both levels of
experimentation.

An important reason for the poor uptake of closed-loop methods in neuroscience  
is that existing implementations are often ad hoc, difficult to install or  
extend, and not integrated with modern pipelines for data analysis or  
sharing. By providing Bonsai users with accessible, open-source, and  
sustainably engineered tools for closed-loop control, we will directly address  
this barrier and enable a new type of experimentation where researchers can  
both read and write the neural code in real time. This will accelerate  
discovery across both basic and translational neuroscience.

Members of Prof.~Stanley’s lab have already prototyped some of their  
closed-loop control methods in Bonsai using our Python scripting
interface\footnote{https://bonsai-rx.org/python-scripting/}.  
In this project, we will collaborate with his group to re-implement these  
methods in C\# for high-performance integration into Bonsai.ML, supported by a  
comprehensive suite of unit tests to ensure functional equivalence with the  
original C++ implementations. This integration will ensure both the scientific  
impact of Prof.~Stanley’s innovations and their long-term sustainability within  
a widely adopted open-source ecosystem.

\subsection*{Governance}

We will create a Bonsai.ML governance structure comprising top-notch
neuroscientists that use Bonsai in their research and are heavily invested on
its future development.
%
This committee will advise us on building a long-term development roadmap for
Bonsai.ML.
