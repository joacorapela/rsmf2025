
\begin{figure}
    \centering
    \includegraphics[width=6in]{activitiesGraphs/activities_larger.png}

    \caption{Proposed activities. The initials below each task name indicate
    the responsible team member (JR: Joaquin Rapela, NG: NeuroGEARS Ltd). See
    text for details.}

\end{figure}

\subsection*{Approach}

\subsubsection*{Documentation}

\paragraph{Summary:} Expand Bonsai.ML documentation.

\paragraph{Previous work:} Bonsai.ML already contains detailed
\href{https://bonsai-rx.org/machinelearning/index.html}{documentation} for all
distributed packages. This documentation includes installation instructions,
API listings, and Bonsai working examples that users can copy and paste into
their Bonsai local runtime and execute. We will improve this documentation as
described next.

\paragraph{Future work (responsible team member):}\mbox{}\\
\begin{itemize}

    \item \textbf{Re-structure documentation following the style of
        \href{https://scikit-learn.org/}{scikit-learn} (NG)}. Our current
        documentation lacks accessible statistical explanations of each method
        and clear links between related methods.  We will adopt the style of
        \href{https://scikit-learn.org/}{scikit-learn}, where methods are
        introduced with high-level descriptions, grouped into categories, and
        cross-referenced to practical examples.

    \item\textbf{Provide more examples (NG)} on the application of the already
        integrated ML methods to new types of behavioural and neural data.
        %
        These examples will be similar to the
        \href{https://bonsai-rx.org/machinelearning/examples/README.html}{existing
        ones}, but will be more detailed.
        %
        They should attract to Bonsai.ML a wider group of experimental
        neuroscientists.

    \item\textbf{Include video tutorials (NG)} on the use of the Bonsai.ML
        package, as done in
        \href{https://mark-kramer.github.io/Case-Studies-Python}{Case Studies
        for Neural Data Analysis}, for example
        \href{https://youtu.be/Oj9e2bB3BfI}{here}.

    \item\textbf{Add documentation for methods developers (NG)} as the current
        one is targeted to Bonsai.ML users. For example, we will provide
        detailed instructions to Python developers on how to integrate their
        methods into Bonsai.ML, including explanations about how handles
        Python's global interpreter lock in C\#.

    \item\textbf{Create case studies (JR)} for intelligent experimental control
        in Bonsai, similar to the
        \href{https://mark-kramer.github.io/Case-Studies-Python/intro.html}{case
        studies for neural data analysis in Python}, or those for
        \href{https://mbmlbook.com/index.html}{model based machine learning},
        by our Microsoft collaborators.

    \item\textbf{Publish a first Bonsai.ML paper (JR)} describing its functionality, as
        companion papers substantially increases the adoption of software
        packages \citep{lopesEtAl15,guilbeaultEtAl21}.

\end{itemize}

\noindent\rule{\textwidth}{1pt}
\subsubsection*{Training}

\paragraph{Summary:} Organise Bonsai course, with a Bonsai.ML module.

\paragraph{Previous work:} Since 2017, NeuroGEARS Ltd has organised at least
two Bonsai courses per year at different universities, and
\href{https://bonsai-rx.org/learn/}{some of them} can be viewed online. We will
deliver a one-week-long Bonsai course, hosted at the Sainsbury Wellcome Centre,
for approximately 20 students, targeted to Bonsai users with intermediate
understanding of the language. The structure of the course will be similar to
previous ones (e.g., \href{https://neurogears.org/st-andrews-2024/}{2024 Bonsai
Course at St.~Andrews University}).

\paragraph{Outputs:} course delivery, online course material (including lecture
slides, worksheets and video recordings).

\paragraph{Responsible team members:} GL, JR.

\noindent\rule{\textwidth}{1pt}
\subsubsection*{Integrations}

\subsubsection*{Probabilistic programming: Infer.NET}

\paragraph{Background:} Most of the probabilistic models currently integrated
into Bonsai are implemented in Python. These serve as excellent demonstrations
of how Python applications can connect to the Bonsai ecosystem, and they are
central to our aim of attracting Python developers to contribute to Bonsai.ML.
%
However, Python implementations are substantially slower than equivalent C\#
code. For demanding real-time applications, C\# implementations are preferable,
especially when expressed in a probabilistic programming language (PPL).

In addition, the existing Python and C\# implementations of probabilistic
models (e.g., linear dynamical systems, hidden Markov models, Bayesian linear
regression — see
\href{https://bonsai-rx.org/machinelearning/examples/README.html}{examples})
are relatively complex and heterogeneous, in the sense that the implementation
of learning and inference in linear dynamical systems is non-trivial and has
little in common with the implementation of learning and inference in Bayesian
linear regression or in hidden Markov models.
%
In contrast, implementations in a C\# PPL would be much simpler, since PPLs
abstract from their users the complexities of their inference algorithms,
leading to sophisticated inferential methods implemented in a few lines of
code.
%
Also, thanks to this abstraction, implementations of different probabilistic
models in PPLs are substantially more homogeneous in PPLs than in general
purpose ones.

Currently, when we decide to incorporate a new probabilistic model into
Bonsai.ML, we need to implement the learning and inference algorithms for the
specific model, which is generally quite complex and time consuming.
%
In contrast, implementing a new probabilistic model in a PPL only requires the
specification of how the model generates observations, without the
need of specific learning or inference algorithms.
%
Thus, integrating a PPL into Bonsai.ML would greatly simplify the addition of
new probabilistic models to the language.

Fortunately, C\# has an excellent PPL:
\href{https://dotnet.github.io/infer/}{Infer.NET}, developed at Microsoft
Research Cambridge since 2004, used in
\href{https://dotnet.github.io/infer/papers.html}{hundreds of papers}, and
\href{https://www.microsoft.com/en-us/research/blog/the-microsoft-infer-net-machine-learning-framework-goes-open-source/}{open-sourced
in 2018}.  Infer.NET uses deterministic approximate inference, enabling fast
and scalable solutions. For
\href{https://www.microsoft.com/en-us/research/blog/the-microsoft-infer-net-machine-learning-framework-goes-open-source/}{example},
it has powered systems that extract knowledge from billions of web pages
(petabyte-scale data) — the kind of scalability critical for real-time
inference in Bonsai.

\paragraph{Previous work:} Our Microsoft collaborator, Dr.~Tom Minka, invented
a seminal algorithm for inference in graphical models, the Expectation
Propagation algorithm~\citep{minka01}, and is the lead developer of Infer.NET.
Please refer to his letter of support.

\paragraph{Tasks:}\mbox{}\\

\begin{description}

    \item[in\_learn:] the responsible team member is skilled in probabilistic
        programming, but not in Infer.NET. He will invest two weeks in learning
        well the language.

    \item[in\_OBLR:]  Implement in Infer.NET the Online Bayesian Linear
        Regression model, currently implemented in C\# in Bonsai.ML. Develop
        test cases to check that the output of the Infer.Net and previous C\#
        implementations are equal.

    \item[in\_LDS:] Implement in Infer.NET the Linear Dynamical Systems model,
        currently implemented in Python. Develop test cases to check that the
        output of the Infer.Net and previous Python implementations are equal.

    \item[in\_HMM:] Implement in Infer.NET the Hidden Markov Model, currently
        implemented in Python. Develop test cases to check that the output of
        the Infer.Net and previous Python implementations are equal.

    \item[in\_PPdecoder:] Implement in Infer.NET the Point Process Decoding
        model, currently implemented in Python. Develop test cases to check
        that the output of the Infer.Net and previous Python implementations
        are equal.

    \item[in\_docs:] For each of the models integrated above, we will add
        extensive documentation explaining how the models were implemented in
        Infer.NET. The aim is to enable Bonsai.ML users to learn from these
        examples and apply the same principles to build and perform inference
        on their own custom probabilistic models, beyond the ones provided in
        Bonsai.ML-Infer.NET.

\end{description}

\paragraph{Impact:}
This integration will accelerate
inference, simplify and standardise inference programs, and empower Bonsai
users to create new probabilistic models and inference algorithms with just a
few lines of code. Consequently, this activity will drastically improve the
maintainability of the software and facilitate the incorporation of new
probabilistic functionality to the Bonsai ecosystem.

\paragraph{Milestones and Indicators:}\mbox{}\\

\begin{description}

    \item[in\_m1:]  implement OBLR in Bonsai.ML.Infer.NET

    \item[in\_i1:] package Bonsai.ML.Infer.NET.OBLR published in nuget

    \item[in\_m2:] implement LDS in Bonsai.ML.Infer.NET

    \item[in\_i2:] package Bonsai.ML.Infer.NET.LDS published in nuget

    \item[in\_m3:] implement HMM in Bonsai.ML.Infer.NET

    \item[in\_i3:] package Bonsai.ML.Infer.NET.HMM published in nuget

    \item[in\_m4:] implement Point Process Decoder in Bonsai.ML.Infer.NET

    \item[in\_i4:] package Bonsai.ML.Infer.NET.PP.decoder published in nuget

    \item[in\_m5:] detailed documentation in repos Bonsai.ML.Infer.NET.OBLR
        Bonsai.ML.Infer.NET.LDS Bonsai.ML.Infer.NET.HMM
        Bonsai.ML.Infer.NET.PP.decoder

    \item[in\_i5:] detailed documentation available in the above repos

    \item[in\_m6:] publish all of the previous Bonsai.ML.Infer.NET.* packages
    \item[in\_i6:] packages Bonsai.ML.Infer.NET.* available in nuget

\end{description}

\paragraph{Responsible team member:} JR.

\subsubsection*{Closed-loop optogenetic control tools: CLOCTools}

\paragraph{Background:} Closed-loop neural control represents a transformative
advance in neuroscience:  rather than delivering stimulation at fixed,
open-loop schedules, it enables precisely timed interventions based on ongoing
brain and behavioural activity. This paradigm allows researchers to move beyond
observing correlations to  directly testing causal mechanisms of neural
dynamics, plasticity, and behaviour. Critically, closed-loop stimulation has
already proved transformative in the  clinic, for example in deep brain
stimulation for Parkinson’s disease and  epilepsy, while its extension to other
disorders (e.g. depression, obsessive  compulsive disorder) remains an active
area of research. Despite this promise, widespread adoption has been limited
by the lack of accessible, well-engineered,  and sustainable software
frameworks for real-time experimental control.

Prof.~Garrett Stanley (Georgia Tech and Emory University, US) is a pioneer in
closed-loop neuroscience, having developed groundbreaking methods that combine
real-time neural control with systems neuroscience. He recently contacted us to
explore integrating their existing
\href{https://cloctools.github.io/}{CLOCTools}, originally implemented in
RTXI/C++, into Bonsai.ML.  This represents a unique opportunity: Bonsai already
excels at real-time closed-loop control in behavioural experiments, and
extending it to include state-of-the-art closed-loop neural control will
position the platform as the first sustainable, general-purpose framework for
both levels of experimentation.

An important reason for the poor uptake of closed-loop methods in neuroscience
could be that existing implementations are often ad hoc, difficult to install
or  extend, and not integrated with software for experimental control. By
providing Bonsai users with accessible, open-source, and sustainably engineered
tools for closed-loop control, we will directly address this barrier and enable
a new type of experimentation where researchers can both read and write the
neural code in real time. This will accelerate discovery across both basic and
translational neuroscience.

\paragraph{Previous work:} Members of Prof.~Stanley’s lab have already
prototyped some of their closed-loop control methods in Bonsai using our
\href{https://bonsai-rx.org/python-scripting/}{Python scripting interface}.
Please refer to the repositories in the
\href{https://github.com/ndac-bonsai}{ndcac-bonsai} organisation containing
these prototypes (ndac stands for neural dynamic adaptive control).

\paragraph{Responsible team members:} JR

\paragraph{Subtasks:}\mbox{}\\

\begin{itemize}

    \item implement in C\# the functionality in the repositories of the
        organization \href{https://github.com/ndac-bonsai}{ndac-bonsai}.

    \item deploy a comprehensive battery of test cases to check the correctness
        of the C\#/Bonsai implementation against the Python one.

    \item extend functionality of Bonsai development environment to allow users
        graphically build probabilistic models.

    \item develop documentation, tutorials and use cases for experimental
        neuroscientists with no training on optimal feedback control.

\end{itemize}

\noindent\rule{\textwidth}{1pt}
\subsubsection*{Governance}

We will create a Bonsai.ML steering committee that will be responsible for
approving project milestones and advise us on building a long-term development
roadmap for Bonsai.ML.
%
Responding to guidance and feedback from the steering committee ensures
Bonsai.ML addressed pressing neuroscience needs on an international scale.

Several renowned experimental and computational neuroscientists around the
world are heavily invested in Bonsai, are very interested in adding ML
functionality to their Bonsai workflows, and have agreed to join the Bonsai.ML
steering committee. We list them below.

\begin{description}

    \item[Prof.~Garrett Stanley,] leader of the Laboratory for the Control of
        Neural Systems, Georgia Tech, US. Expert on close-loop control of
        neurophysiological systems. He contacted us to migrate to Bonsai
        a package for real-time cortical state estimation and the CLOCTools
        referred above, both originally written in C++/RTXI.

    \item[Prof.~Aman Saleem,] director of the Saleem Lab at the Institute for
        Behavioural Neuroscience, University College London. Prof.~Saleem is the
        author of \href{https://bonvision.github.io/}{Bon-Vision}, a software
        package that creates and controls visual environments in close loop,
        built on top of Bonsai.

    \item[Prof.~Josh Siegle,] senior scientist and lead of the
        electrophysiology group at the Allen Institute for Neural Dynamics. He
        leads the development of Open Ephys GUI, which is tightly integrated
        with Bonsai.

    \item[Prof.~Ken Harris,] co-director of the Cortexlab, University College
        London, and founding director of the International Brain Laboratory,
        that uses Bonsai for reproducible experimental control across 22
        laboratories around the world.

    \item[Prof.~Athena Akrami,] director of the Learning, Inference and Memory
        Lab, at the SWC, that uses Bonsai extensively for the control of
        complex rodent experiments in her lab.

\end{description}

\noindent\rule{\textwidth}{1pt}
\subsection*{Management}

We will conduct management activities at different frequencies:

\begin{description}

    \item[Twice a year:] we will convene the steering committee. Two weeks in
        advance of each meeting, we will circulate a progress report
        summarising achieved milestones, proposed future activities, and the
        most recent version of the Bonsai.ML long-term roadmap.
        %
        The steering committee will review the report, endorse completed
        milestones, and provide feedback or suggestions on planned activities.
        %
        Based on project progress and committee input, we will revise the
        long-term roadmap, ensuring that Bonsai.ML development remains aligned
        with community needs and strategic goals.
        %
        Minutes and key decisions from these meetings will be documented and,
        where possible, shared with the Research Software Maintenance Fund
        through appropriate reporting channels.

    \item[Every month:] we will hold meetings between the project lead, the
        project co-lead, the RSE, and the external project co-lead, to evaluate
        the project progress.

    \item[Weekly:] as has been our practice since the start of the Bonsai.ML
        project, the RSEs will meet with the external project co-lead to
        discuss issues that appeared during the week, review activities for the
        following week, and adjust project directions.

\end{description}

Meetings with collaborators will be arranged as needed.
%
At the SWC, GCNU and NG we are experimental and computational neuroscientists
with successful collaborative experience, and we have no doubt that the
proposed collaborations will be of the same kind,
%
specially since we have successfully interacted in the past with most of the
propose collaborators.
%
Please refer to their letters of support.
