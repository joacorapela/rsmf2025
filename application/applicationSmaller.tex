
\documentclass[12pt]{article}

\usepackage{xcolor}
\usepackage{tcolorbox}
\usepackage{verbatim}

\newenvironment{instruction}{%
    \begin{tcolorbox}[colback=red!5,colframe=red,title=Instruction]%
}{%
    \end{tcolorbox}%
}

\title{Consolidating Bonsai as a Standard for Neuroscience Intelligent
Experimental Control}

\author{Joaqu\'{i}n Rapela}

\begin{document}

\maketitle

\section*{Summary}

\begin{instruction}

Please provide a summary in plain English of your proposed work.

This summary will be made publicly available on external-facing websites, therefore do not include any confidential or sensitive information.

Please note that by submitting an application, you consent to the information provided on this page (Project Summary, UKRI areas) being publicly disseminated. This includes information from both successful and unsuccessful applications. Information supplied in the other sections of the application will not be published.

It may also be used to help identify suitable reviewers.

\end{instruction}

\subsection*{Summary (500 words)}

\pagebreak

\section*{Core Team}

\begin{instruction}

Tell us who will deliver the proposed work. Create an entry for each core team member.

This section is for naming the people who will make key contributions to the work.

Any unnamed role resource, such as a research software engineer role where an individual has not yet been identified, should be included and justified in the Resources section. All project leads (including international ones, as well as co-leads) need to be named.

If a single person has multiple roles, select the main one under "Role"
There can only be one project lead, and they must be based at an eligible UK Research Organisation. The project lead should normally be the person submitting the application.
If the project lead has changed since the EoI, please contact us to have an invitation to the new lead issued.

\end{instruction}

\pagebreak

\section*{Software}

\begin{instruction}

Please provide details of the software to be supported.

This information may be used by reviewers as part of the assessment of your application. It may also be used to provide an anonymised and aggregated summary across all applications.

\end{instruction}

    \begin{description}

        \item[a. Software to be maintained:]

        \item[b. Code repository (optional):]

        \item[c. Website (optional):]

        \item[d. Year development started:]

        \item[e. Year of first release:]

        \item[f. Programming language:]

    \end{description}

\pagebreak

\section*{Vision}

\begin{instruction}

Please describe what you are hoping to achieve with this funding.

Discuss both your vision and objectives for the technical development, as well as your plans to improve your software's sustainability and support EDIA considerations.

\end{instruction}

\subsection*{Vision (300 words)}

\begin{instruction}

Explain how your proposed work will:

\begin{itemize}
    \item have a benefit and impact on research, with specific examples of research impact in the UK
    \item build the community around the software
    \item increase the adoption of the software
    \item improve the maintainability of the software
    \item advance good practice
\end{itemize}

\end{instruction}

\subsection*{Objectives (200 words)}

\begin{instruction}

Clearly describe the aims and objectives of this work.

\end{instruction}

\subsection*{Timeliness and Sustainability (300 words)}

\begin{instruction}

Justify why it is important that your project is funded in this round, and include:

    \begin{itemize}
        \item why it is important that this work is funded now, rather than at a different time
        \item what other funding sources have been investigated, and why these are not suitable
        \item how the software will be sustained and maintained after the RSMF funding ends, and how the RSMF funding will help achieve this
    \end{itemize}

\end{instruction}

\subsection*{EDIA (300 words)}

\begin{instruction}
Explain how you will embed equity, diversity, inclusivity and accessibility considerations into your proposed work and the software being maintained, and how these will guide your aims, objectives, activities and outputs.

This can include (but is not limited to) considerations applying to your team, your community, and your software.
\end{instruction}

\pagebreak

\section*{Approach (2000 words)}

\subsection*{Approach}

\begin{instruction}

Explain what work you have planned, and how you will manage this. Cover both the technical as well as project management areas.

Planned work

    \begin{itemize}
        \item where applicable: summary of any previous work and how this will be built upon and progressed
        \item where applicable: a clear and transparent methodology
        \item effective and appropriate activities to achieve your objectives
        \item which team members are responsible for each activity and task
        \item expected outputs for each task and how they relate to each other
        \item a timeline with a feasible workplan
        \item strategy to maximise translation of outputs into outcomes and impacts
    \end{itemize}

Management

    \begin{itemize}
        \item how the work will be managed and progress monitored and evaluated
        \item how the current software project governance and infrastructure will contribute to the success of the work
        \item what risks to delivery were identified and how they will be managed
    \end{itemize}


\end{instruction}

\subsection*{Workplan (optional)}

\begin{instruction}
You may choose to upload a one page document such as a Gantt chart to help illustrate timing of and links among the activities.
\end{instruction}

\pagebreak

\section*{Capability to deliver (1000 words)}

\begin{instruction}
Provide evidence of how you and your team have:

    \begin{itemize}
        \item the relevant experience (appropriate to career stage) to deliver the objectives
        \item the right balance of skills and expertise to cover the proposed work
        \item the appropriate leadership and management skills to ensure delivery
        \item contributed to developing good practice in your communities
    \end{itemize}

Where applicable, discuss your approach to developing others.

\end{instruction}

\pagebreak

\section*{Project partners}

\begin{instruction}

A project partner is a collaborating organisation that plays an integral role in the proposed work. You should describe the nature of this support in the Approach section of your application.

Project partners contribute to the delivery of the project and should not normally request funding from the grant. However, travel and subsistence costs incurred by the lead organisation to enable project partner involvement may be included—these must be fully justified in the Resources section.

You cannot include an individual as an applicant (i.e., project lead, co-lead, or any Core Team role) if they, or their organisation, are named as the project partner contact.

All project partners must be listed as contributors below, and a letter of support for each must be uploaded as a single combined PDF.

\end{instruction}

\subsection*{Letter(s) of Support (optional)}

\pagebreak

\section*{Resources}

\begin{instruction}

Please provide details of the funding requested.

Use the breakdown categories listed in this section, and discuss the main resource requirements.

You will also need to upload a consolidated budget from the lead organisation using a standard FEC costing format.

\end{instruction}

\subsection*{Justification of resources (1000 words)}

\begin{instruction}

Justify the application’s more costly resources, in particular:

\begin{itemize}
    \item any staff costs
    \item significant costs related to collaboration or community engagement
    \item any consumables beyond typical requirements
    \item infrastructure costs
    \item all resources that have been costed as ‘Exceptions’
\end{itemize}

You do not need to justify Estates and Indirect costs.

We are not looking for a detailed breakdown of each cost, but want you to demonstrate how the resources you are applying for are comprehensive, appropriate and justified, and represent the optimal use of resources to achieve the intended outcomes.

\end{instruction}

\subsubsection*{Total funding requested}

\subsubsection*{Directly incurred - Staff}

\subsubsection*{Directly incurred - Travel and Subsistence}

\subsubsection*{Directly incurred - Other}

\subsubsection*{Directly allocated - Staff}

\subsubsection*{Directly allocated - Estates}

\subsubsection*{Directly allocated - Other}

\subsubsection*{Indirects}

\subsubsection*{Exceptions}

\subsubsection*{Budget}

The lead organisation's consolidated budget in a standard FEC costing format.

\end{document}
