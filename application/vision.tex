More than ten years ago our business partner, Dr.~Goncalo Lopes, created
Bonsai\footnote[1]{\url{https://bonsai-rx.org/}}, a reactive visual programming
language most widely used for neuroscience experiment control.
%
Because it is based on the reactive programming
framework~\footnote[2]{\url{https://introtorx.com/}}, Bonsai uniquely deals
with asynchronous events,
%
and because it is a visual programming language it allows people with no
programming experience to quickly develop high-performance data acquisition and
experimental control systems.
%
Importantly, Bonsai can interact with a large set of hardware and software
packages used by the experimental neuroscience community (e.g., Arduino, Open
ephys, Neurophotometrics). It has been adopted by thousands of users in the UK
and all around the world (7,000 downloads per year and 1,000 citations per year
of the core Bonsai paper~\citep{lopesEtAl15}).

Over the past decade, machine learning (ML) has undergone a transformative
revolution — beginning with breakthroughs in computer vision, advancing through
natural language processing, and more recently progressing toward
general-purpose intelligence in large foundational models.
%
At the Gatsby Unit we are shaping this revolution.

ML has become essential across most branches of science — neuroscience in
particular. Yet, many neuroscience experiments are still controlled by simple,
rigid rules (e.g., delivering a reward only when a rat pokes left, but not
right).

In 2022, we recognized that integrating ML into the Bonsai platform could
enable a radically new form of intelligent experimental control. With this
capability, scientists could, for example, infer from neural recordings when a
rat is about to poke left, and deliver a reward without waiting for the
physical action.
%
To realise this vision, sponsored by
BBSRC\footnote[3]{\url{https://gow.bbsrc.ukri.org/grants/AwardDetails.aspx?FundingReference=BB\%2FW019132\%2F1}}
we developed
Bonsai.ML\footnote[4]{\url{https://bonsai-rx.org/machinelearning}}, a package
that brings ML functionality into the Bonsai ecosystem.

Since most experimental neuroscientists currently using Bonsai are not highly
skilled in machine learning, to maximise the impact of Bonsai.ML, we need to
invest extra efforts on documentation, training and community building, as we
describe in the next section.

The proposed documentation and dissemination activities will help users of
Bonsai.ML better understand the ML methods in the package, perform more
sophisticated neuroscience experiments, and produce unprecedented new
neuroscientific findings.
%
In addition, these activities will attract to Bonsai new experimental
neuroscience users interested in adding ML functionality to their experiments,
and create a new Bonsai community of ML methods developers, as we
explain below.
%
We have focused this proposal on applications of Bonsai.ML to experimental
neuroscience, since this is our area of expertise. However, Bonsai is used in
other experimental domains, like live exhibitions and robotics, where Bonsai.ML
should also be relevant.

Bonsai has already transformed experimental design, allowing researchers to
perform studies of complexity previously unimaginable. With Bonsai.ML, we aim
to spark a second revolution — empowering scientists to control experiments in
intelligent, adaptive ways they had never thought possible.

Most current ML methods are designed to operate offline, with datasets stored
on disk, after data collection has finished.
%
Bonsai requires ML methods that can process online data, while data is being
collected, and in a close-loop manner.
%
This has two important implications.
%
First, Bonsai.ML provides a new type of ML methods for real-time neuroscience
data, operating in in close loop. Used by experimental neuroscientists, these
method could generate unprecedented findings on brain function.
%
Second, Bonsai, as an excellent source of real-time neural and behavioural data,
could become of interest to ML methods developers wanting to apply their
methods to real-time data in close loop.
%
Therefore, the dissemination of Bonsai.ML among machine learning methods
developers could attract to Bonsai a new community of ML methods developers.
