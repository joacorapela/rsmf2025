\subsection*{Bonsai}

Bonsai\footnote[1]{\url{https://bonsai-rx.org/}} is a software ecosystem used by thousands
of (mostly experimental neuroscience) users in the UK and all around the world
(reference usage numbers: number of downloads, forum members, distribution
around the world).
%
Being a visual-programming language, Bonsai allows scientists with little
programming experience to control sophisticated neuroscience experiments.
%
Bonsai is the centre of a growing ecosystem of open source hardware and
software \ldots


\subsection*{Background on Bonsai.ML}

Machine learning (ML) is now essential for most branches of science,
neuroscience in particular.
%
A few years back, ML was mostly used by statisticians, but today it is becoming
a popular technology, thanks to many ML libraries (like PyTorch, GPyTorch,
SickLearn) that simplify its use.
%
Yet, current neuroscience experiments are still controlled by simple means
(e.g., deliver a reward when a rat pokes left but not right).
%
In 2022, we realised that adding ML functionality to Bonsai could empower
Bonsai users and enable a radically new type of intelligent experimental
control.
%
% Experimental neuroscientists with machine learning tools could create
% unprecedented experiments unlocking unsolved mysteries of brain function.
%
We created the
Bonsai.ML\footnote[2]{\url{https://bonsai-rx.org/machinelearning}}
package providing machine learning functionality to the Bonsai ecosystem.

\subsection*{Machine learning functionality in Bonsai}

Bonsai is integrated with two popular methods for animal pose tracking:
DeeplabCut\footnote[3]{\url{https://github.com/bonsai-rx/deeplabcut}} and
SLEAP\footnote[4]{\url{https://bonsai-rx.org/sleap/}}.
%
We have added to Bonsai Bayesian Linear Regression models, Linear Dynamical
Systems (LDS), Hidden Markov Models (HMM) and point-process decoders.
%
We have used these models to characterise a diverse range of behavioural and
neural recordings.
%
All methods we have developed had been demonstrated using behavioural and
neural
data\footnote[5]{\url{https://bonsai-rx.org/machinelearning/examples/README.html}}.

Bonsai is implemented in C\#, while most machine learning models are
implemented in other languages, like Python, Julia or R.
%
To facilitate the incorporation of machine learning functionality into Bonsai,
and motivate more ML methods developers to contribute their methods, we
developed the Bonsai-Python
Scripting\footnote[6]{\url{https://bonsai-rx.org/python-scripting/}}
package and used it to interface Bonsai with ML packages written in Python.

Last December we organised the first Bonsai Developers conference at
UCL\footnote[7]{\url{https://conference.bonsai-rx.org/2024/}}. It was attended by more
than 35 participants from all over the world, including leading neuroscientists
from the Allen Institute for Brain Dynamics, Janelia Research Campus, the
Champalimaud Centre for the Unknown and University College London. We presented
the Bonsai.ML project, which was received with excitement by the Bonsai
community.

\subsection*{Proposed activities}

The potential of providing machine learning functionality to experimental
neuroscientists is very large, as it could help them create unprecedented
experiments leading to groundbreaking scientific findings.
%
We have created a first version of a Bonsai package that could be
transformative for experimental control.
%
However, since most experimental neuroscientists currently using Bonsai are not
highly skilled in ML, to maximise the impact of Bonsai.ML, we need to invest
extra efforts on documentation, training and community building, as we propose
below.

The vast majority of current ML methods process data in batch form, after
experiments are finished. Bonsai is a unique source of real-time behavioural and
neural data, it is best suited for real-time ML methods, and should be a data
source of great interest to the growing community of real-time ML methods
developers. Thus, our community building activities will attempt to expand
Bonsai's current user base of experimental neuroscientists with ML methods
developers.

\subsubsection*{Documentation}

The current documentation of
Bonsai.ML\footnote[8]{\url{https://bonsai-rx.org/machinelearning/index.html}}
is good; it describes the API and demonstrates the usage of packages for
processing behavioural and neural data. As part of this project we will improve
this documentation by:

\begin{enumerate}

    \item providing more detailed examples on the application of the
        already integrated ML methods to new types of behavioural and
        neural data.

    \item including video tutorials on the use of the Bonsai.ML
        package.

    \item adding documentation for methods developers. The current
        documentation is targeted to experimental neuroscientists. For methods
        developers we will add detailed examples and video tutorials on the
        implementation of machine learning methods using the reactive framework
        in C\#, and on how to integrate existing ML packages, written in
        other programming languages (e.g., Python or Julia), into Bonsai.

    \item crucial to Bonsai.ML is to provide its users with comprehensive
        information about how the ML methods in the package operate, what their
        assumptions are, and what the generated outputs mean. We will use
        Bonsai.ML to provide training to its users on machine learning.  An
        excellent resource exist presenting case studies for neural data
        analysis in
        Python\footnote[9]{\url{https://mark-kramer.github.io/Case-Studies-Python/intro.html}}.
        We will create new case studies for intelligent experimental control in
        Bonsai.

    \item we have learnt that publishing papers substantially increases the
        adoption of packages \citep{lopesEtAl15,guilbeaultEtAl21}. To expand
        the user base of Bonsai.ML, we will publish a paper describing its
        operation.

\end{enumerate}

\subsubsection*{Training}

NeuroGEARS Ltd (the non-profit organisation that is the main contributor to the
development of Bonsai) has organised xxx Bonsai course at different
universities, and some of them can be viewed
online\footnote[10]{\url{https://bonsai-rx.org/learn/}}. As part of the
proposed project we will organise a new Bonsai course with two tracks, one for
experimental neuroscientists and the second one for methods developers. The
course will start with an introduction to Bonsai common to both tracks. The
experimental track will teach how to use Bonsai.ML packages. The developers
track will teach the reactive programming framework and how to use different
programming languages to interface with Bonsai.

\subsubsection*{Community}

We hope that in the near future experimental neuroscientists will use methods
provided in Bonsai.ML to control their experiments, while methods developers
will integrate their methods into Bonsai following provided examples. To
accelerate this adoption, we will initially collaborate with experimental
neuroscientists and help them use tools provided in Bonsai.ML to control their
experiments, and we will collaborate with methods developers to help them
integrate their methods into Bosai.ML.

\paragraph{Collaborations with experimental neuroscientists:} Several
Neuroscience institutions around the world are using Bonsai for experimental
control: the Allen Institute for Neural Dynamics and Janelia Research Campus in
the US, the Chamapalimaud Centre for the Unknown in Portugal, the Sainsbury
Wellcome Centre and the Institute for Behavioural Neuroscience, both at
University College London, among others. We will collaborate with experimental
neuroscientists at the Sainsbury Wellcome Centre and at the Allen Institute
for Neural Dynamics on the integration of machine learning functionality into
their experiments.

\paragraph{Collaborations with methods developers;} Recently Prof.~Garrett
Stanley, director of the Laboratory for the Control of Neural
Systems\footnote[11]{\url{https://stanley.gatech.edu/}}, at Georgia Institute
of Technology, contacted us asking for assistance in integrating into Bonsai.ML
functionality for close-loop neural control, that his laboratory had developed
in RTXI/C++. We will assist his laboratory on this integration.

%
% We will also collaborate with Prof.~Matthias
% Hennig\footnote[12]{\url{https://homepages.inf.ed.ac.uk/mhennig/index.html}}
% on the integration of online spike sorting routines developed by his group
% into Bonsai.ML.

Bonsai is already part of ONIX system for long-duration, low-latency and
high-throughput neural data
acquisition\footnote[13]{\url{https://www.nature.com/articles/s41592-024-02521-1}}.  We
will evaluate real-time neural data processing functionality already developed
in Bonsai.ML with ONIX recordings and create Open Ephys
plugins\footnote[14]{\url{https://open-ephys.github.io/gui-docs/User-Manual/Plugins/}} to
allow to easily use Bonsai.ML with ONIX.

\subsubsection*{Governance}

Bonsai.ML currently has no governance structure, which is a severe limitation
considering the relevance of both experimental and computational
neuroscientists to guide its development.
%
We will create this structure comprising top-notch neuroscientists that use
Bonsai in their research and are heavily invested on its future development.
%
This committee will advise us on building a development roadmap for Bonsai.ML.
It will include:

\begin{description}

    \item[Prof.~Thomas Mrsic Flogel:] Director of the Sainsbury Wellcome
        Centre, University College London, where Bonsai is the standard
        platform for experimental control, and research lead on the BBSRC grant
        that funded the creation of the Bonsai.ML package.

    \item[Prof.~Maneesh Sahani:] Director of the Gatsby Computational
        Neuroscience Unit, University College London, and research co-lead on
        the BBSRC grant that funded the creation of the Bonsai.ML package.

    \item[Prof.~Aman Saleem:] Director of the Saleem lab, University College
        London, and creator of the Bonsai Bon Vision package.

    \item[Prof.~Kenneth Harris:] Co-director of the Cortexlab, University
        College London, and founding member of the International Brain
        Laboratory, where he has played a pivotal role in developing the
        project's data architecture. IBL uses Bonsai for experimental control.

    \item[Prof.~Garrett Stanley:] Director of the Laboratory for the Control of
        Neural Systems, Georgia Institute of Technology. He is now migrating to
        Bonsai software packages for the close-loop control of neural systems
        his group had developed in the past using RTXI.

    \item[Dr.~Jakob Voigts:] Director of the Voigts Lab, HHMI's Janelia
        Research Campus, co-founder of Open Ephys and senior author of ONIX,
        which uses Bonsai as experimental control platform.

    \item[Dr.~Josh Siegle:] Director of the Electrophysioloy group at the Allen
        Institute for Neural Dynamics, co-founder of Open Ephys and author of
        ONIX.
        
\end{description}
