% Bonsai

Bonsai\footnote[1]{https://bonsai-rx.org/} is a software ecosystem used by thousands
of (mostly experimental neuroscience) users in the UK and all around the world
(reference usage numbers: number of downloads, forum members, distribution
around the world).
%
Being a visual-programming language, Bonsai allows scientists with little
programming experience to control sophisticated neuroscience experiments.
%
Bonsai is the centre of a growing ecosystem of open source hardware and
software \ldots


% Bonsai.ML and the importance of providing ML functionality to experimental
% neurosciencetists

Machine learning (ML) is now becoming essential for most branches of science,
neuroscience in particular.
%
A few years back, ML was mostly used by statisticians, but today it is becoming
a popular technology, thanks to many ML libraries (like PyTorch, GPyTorch, SickLearn)
that simplify its use.
%
Yet, current neuroscience experiments are still controlled by simple means
(e.g., deliver a reward when a rat pokes left but not right).
%
In 2022, we realised that adding ML functionality to Bonsai could empower
Bonsai users and enable a radically new type of intelligent experimental
control.
%
% Experimental neuroscientists with machine learning tools could create
% unprecedented experiments unlocking unsolved mysteries of brain function.
%
We created the Bonsai.ML\footnote[2]{https://bonsai-rx.org/machinelearning}
package providing machine learning functionality to the Bonsai ecosystem.

% What we did

We have added to Bonsai Bayesian Linear Regression models, Linear Dynamical
Systems (LDS), Hidden Markov Models (HMM) and point-process decoders.
%
We have used these models to characterise a diverse range of behavioural and
neural recordings.
%
All methods we have developed had been demonstrated using behavioural and
neural data\footnote[3]{https://bonsai-rx.org/machinelearning/examples/README.html}.

Bonsai is implemented in C\#, while most machine learning models are
implemented in other languages, like Python, Julia or R.
%
To facilitate the incorporation of machine learning functionality into Bonsai,
and motivate more ML methods developers to contribute their methods, we
developed the Bonsai-Python
Scripting\footnote[4]{https://bonsai-rx.org/python-scripting/}
package and used it to interface Bonsai with ML packages written in Python.

% What we propose to do

The potential of providing machine learning functionality to experimental
neuroscientists with little programming experience is large, as it could help
them create unprecedented experiments leading to groundbreaking scientific
findings.
%
However, to maximise the impact of this new functionality we, as methods
developers, need to invest extra efforts on documentation, training and
community building, as we propose below.

The vast majority of current ML methods process data in batch form, after
experiments are finished. Bonsai is a unique source of real-time behavioural and
neural data, it is best suited for real-time ML methods, and should be a data
source of great interest to the growing community of real-time ML methods
developers. Thus, our community building activities will attempt to expand
Bonsai's current user base of experimental neuroscientists with ML methods
developers.

\begin{description}

    \item[Documentation:] the current documentation of
        Bonsai.ML\footnote[5]{https://bonsai-rx.org/machinelearning/index.html}
        is good; it describes the API and demonstrates the usage of packages
        for processing behavioural and neural data. As part of this project we
        will improve this documentation by:

    \begin{enumerate}

        \item providing more examples on the application of the already
            integrated ML methods to new types of behavioural and neural data.

        \item including include video tutorials on the use of the Bonsai.ML
            packages.

        \item adding documentation for methods developers. The current
            documentation is focused on experimental neuroscientists. For
            methods developer we will add examples and video tutorials on how
            to integrate existing ML packages, written in different programming
            languages, into the Bonsai ecosystem.

        \item crucial to Bonsai.ML is to provide its users with comprehensive
            information about what the ML methods in the package operate, what
            their assumptions are, and what the generated outputs mean. We will
            use Bonsai.ML to provide training to its users on machine learning.
            An excellent resource exist presenting case studies for neural data
            analysis with
            Python\footnote[6]{https://mark-kramer.github.io/Case-Studies-Python/intro.html}.
            We will create case studies for intelligent experimental control
            with Bonsai.ML.

    \end{enumerate}

    \item[Training:] NeuroGEARS Ltd (the non-profit organisation that is the
        main contributor to the development of Bonsai) has organised xxx Bonsai
        course at different universities, and some of them can be viewed
        online\footnote[7]{https://bonsai-rx.org/learn/}. As part of the
        proposed project we will organise a new Bonsai course with two tracks,
        one for experimental neuroscientists and the second one for methods
        developers. The course will start with an introduction to Bonsai common
        to both tracks. The experimental track will teach how to use Bonsai.ML
        packages. The developers track will teach the reactive programming
        framework and how to use different programming languages to interface
        with Bonsai.

    \item[Community:] It is important that we integrated into Bonsai ML methods
        that are used widely by experimental neuroscientists. It is also
        important that the ML functionality we provide satisfies the need of
        these scientists. For these reasons it is vital that we collaborate
        with experimental neuroscientists on the integration of ML methods into
        Bonsai.

        Several Neuroscience institutions around the world are using Bonsai for
        experimental control: the Allen Institute for Neural Dynamics and
        Janelia Research Campus in the US, the Chamapalimaud Centre for the
        Unknown in Portugal, the Sainsbury Wellcome  Centre in London, among
        others. We collaborate with experimental neuroscientists at the
        Sainsbury Wellcomce Centre and at the Sainsbury Wellcome Centre on the
        integration of machine learning functionality into Bonsai.

\end{description}

