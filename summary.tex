\subsection*{Bonsai}

Bonsai\footnote[1]{\url{https://bonsai-rx.org/}} is a software ecosystem used by thousands
of (mostly experimental neuroscience) users in the UK and all around the world
(reference usage numbers: number of downloads, forum members, distribution
around the world).
%
Being a visual-programming language, Bonsai allows scientists with little
programming experience to control sophisticated neuroscience experiments.
%
Bonsai is the centre of a growing ecosystem of open source hardware and
software \ldots


\subsection*{Background on Bonsai.ML}

Machine learning (ML) is now essential for most branches of science,
neuroscience in particular.
%
A few years back, ML was mostly used by statisticians, but today it is becoming
a popular technology, thanks to many ML libraries (like PyTorch, GPyTorch,
SickLearn) that simplify its use.
%
Yet, current neuroscience experiments are still controlled by simple means
(e.g., deliver a reward when a rat pokes left but not right).
%
In 2022, we realised that adding ML functionality to Bonsai could empower
Bonsai users and enable a radically new type of intelligent experimental
control.
%
% Experimental neuroscientists with machine learning tools could create
% unprecedented experiments unlocking unsolved mysteries of brain function.
%
We created the
Bonsai.ML\footnote[2]{\url{https://bonsai-rx.org/machinelearning}}
package providing machine learning functionality to the Bonsai ecosystem.

\subsection*{Machine learning functionality in Bonsai}

Bonsai is integrated with two popular methods for animal pose tracking:
DeeplabCut\footnote[3]{\url{https://github.com/bonsai-rx/deeplabcut}} and
SLEAP\footnote[4]{\url{https://bonsai-rx.org/sleap/}}.
%
We have added to Bonsai Bayesian Linear Regression models, Linear Dynamical
Systems (LDS), Hidden Markov Models (HMM) and point-process decoders.
%
We have used these models to characterise a diverse range of behavioural and
neural recordings.
%
All methods we have developed had been demonstrated using behavioural and
neural
data\footnote[5]{\url{https://bonsai-rx.org/machinelearning/examples/README.html}}.

Bonsai is implemented in C\#, while most machine learning models are
implemented in other languages, like Python, Julia or R.
%
To facilitate the incorporation of machine learning functionality into Bonsai,
and motivate more ML methods developers to contribute their methods, we
developed the Bonsai-Python
Scripting\footnote[6]{\url{https://bonsai-rx.org/python-scripting/}}
package and used it to interface Bonsai with ML packages written in Python.

Last December we organized the first Bonsai Developers conference at
UCL\footnote[7]{\url{https://conference.bonsai-rx.org/2024/}}. It was attended by more
than 35 participants from all over the world, including leading neuroscientists
from the Allen Institute for Brain Dynamics, Janelia Research Campus, the
Champalimaud Centre for the Unknown and University College London. We presented
the Bonsai.ML project, which was received with excitement by the Bonsai
community.

\subsection*{Proposed activities}

The potential of providing machine learning functionality to experimental
neuroscientists is very large, as it could help them create unprecedented
experiments leading to groundbreaking scientific findings.
%
We have created a first version of a Bonsai package that could be
transformative for experimental control.
%
However, since most experimental neuroscientists currently using Bonsai are not
highly skilled in ML, to maximise the impact of Bonsai.ML, we need to invest
extra efforts on documentation, training and community building, as we propose
below.

The vast majority of current ML methods process data in batch form, after
experiments are finished. Bonsai is a unique source of real-time behavioural and
neural data, it is best suited for real-time ML methods, and should be a data
source of great interest to the growing community of real-time ML methods
developers. Thus, our community building activities will attempt to expand
Bonsai's current user base of experimental neuroscientists with ML methods
developers.

\begin{description}

    \item[Documentation:] the current documentation of
        Bonsai.ML\footnote[8]{\url{https://bonsai-rx.org/machinelearning/index.html}}
        is good; it describes the API and demonstrates the usage of packages
        for processing behavioural and neural data. As part of this project we
        will improve this documentation by:

    \begin{enumerate}

        \item providing more detailed examples on the application of the
            already integrated ML methods to new types of behavioural and
            neural data.

        \item including include video tutorials on the use of the Bonsai.ML
            packages.

        \item adding documentation for methods developers. The current
            documentation is targeted to experimental neuroscientists. For
            methods developers we will add detailed examples and video
            tutorials on how to integrate existing ML packages, written in
            different programming languages, into the Bonsai ecosystem.

        \item crucial to Bonsai.ML is to provide its users with comprehensive
            information about how the ML methods in the package operate, what
            their assumptions are, and what the generated outputs mean. We will
            use Bonsai.ML to provide training to its users on machine learning.
            An excellent resource exist presenting case studies for neural data
            analysis in
            Python\footnote[9]{\url{https://mark-kramer.github.io/Case-Studies-Python/intro.html}}.
            We will create new case studies for intelligent experimental
            control in Bonsai.

    \end{enumerate}

    \item[Training:] NeuroGEARS Ltd (the non-profit organisation that is the
        main contributor to the development of Bonsai) has organised xxx Bonsai
        course at different universities, and some of them can be viewed
        online\footnote[10]{\url{https://bonsai-rx.org/learn/}}. As part of the
        proposed project we will organise a new Bonsai course with two tracks,
        one for experimental neuroscientists and the second one for methods
        developers. The course will start with an introduction to Bonsai common
        to both tracks. The experimental track will teach how to use Bonsai.ML
        packages. The developers track will teach the reactive programming
        framework and how to use different programming languages to interface
        with Bonsai.

    \item[Community:] We hope that in the near future experimental neuroscientists
        will use methods provided in Bonsai.ML to control
        their experiments, while methods developers will integrate
        their methods into Bonsai following provided examples. To accelerate
        this adoption, we will initially collaborate with experimental neuroscientists
        and help them use tools provided in Bonsai.ML to control their
        experiments, and we will collaborate with methods developers to help
        the integrate their methods into Bosai.ML.

        Several Neuroscience institutions around the world are using Bonsai for
        experimental control: the Allen Institute for Neural Dynamics and
        Janelia Research Campus in the US, the Chamapalimaud Centre for the
        Unknown in Portugal, the Sainsbury Wellcome  Centre and the Institute
        for Behavioral Neuroscience, both at University College London, among
        others. We will collaborate with experimental neuroscientists at the
        Sainsbury Wellcomce Centre and at the Allen Institute for Neural
        Dynamics on the integration of machine learning functionality into
        their experiments.

        Recently Prof.~Garrett Stanley, director of the Laboratory for the
        Control of Neural
        Systems\footnote[11]{\url{https://stanley.gatech.edu/}}, at Georgia
        Institute of Technology, contacted us asking for assistance in
        integrating into Bonsai.ML functionality for close-loop neural control,
        that his laboratory had developed in RXTI/C++. We will assist his
        laboratory on this integration.
        %
        We will also collaborate with Prof.~Matthias
        Hennig\footnote[12]{\url{https://homepages.inf.ed.ac.uk/mhennig/index.html}}
        on the integration of online spike sorting routines developed by his
        group into Bonsai.ML.

\end{description}

