Bonsai\footnote[1]{\url{https://bonsai-rx.org/}} is a software ecosystem used by thousands
of (mostly experimental neuroscience) users in the UK and all around the world
(7,000 downloads per year and 1,000 citations per year of the core Bonsai
paper~\citep{lopesEtAl15})
%
Being a visual-programming language, Bonsai allows scientists with little
programming experience to control sophisticated neuroscience experiments.

Machine learning (ML) is now essential for most branches of science,
neuroscience in particular.
%
Yet, current neuroscience experiments are still controlled by simple means
(e.g., deliver a reward when a rat pokes left but not right).
%
In 2022, we realised that adding ML functionality to Bonsai could empower
Bonsai users and enable a radically new type of intelligent experimental
control.
%
We created the
Bonsai.ML\footnote[2]{\url{https://bonsai-rx.org/machinelearning}}
package providing machine learning functionality to the Bonsai ecosystem.

Since most experimental neuroscientists currently using Bonsai are not highly
skilled in ML, to maximise the impact of Bonsai.ML, we need to invest extra
efforts on documentation, training and community building, as we propose below.

\subsection*{Documentation}

We will improve the current Bonsai.ML
documentation\footnote[3]{\url{https://bonsai-rx.org/machinelearning/index.html}}
by:

\begin{enumerate}

    \item providing more detailed examples on the application of the
        already integrated ML methods to new types of behavioural and
        neural data.

    \item including video tutorials on the use of the Bonsai.ML
        package.

    \item adding documentation for methods developers, as the current
        documentation is targeted to experimental neuroscientists.

    \item we will use Bonsai.ML to provide training on machine learning. We
        will create case studies for intelligent experimental control in
        Bonsai, as in the case studies for neural data analysis in
        Python\footnote[4]{\url{https://mark-kramer.github.io/Case-Studies-Python/intro.html}}.

    \item we have learnt that publishing papers substantially increases the
        adoption of packages \citep{lopesEtAl15,guilbeaultEtAl21}. To expand
        the user base of Bonsai.ML, we will publish a paper describing its
        operation.

\end{enumerate}

\subsection*{Training}

Since 2017, NeuroGEARS Ltd (the non-profit organisation that is the main
contributor to the development of Bonsai) has organised at least two Bonsai
course per year at different universities, and some of them can be viewed
online\footnote[5]{\url{https://bonsai-rx.org/learn/}}. We will organise a new
Bonsai course with two tracks, one for experimental neuroscientists and the
second one for methods developers.

\subsection*{Community}

To accelerate the adoption of Bonsai.ML, we will collaborate with experimental
neuroscientists and with methods developers.

\paragraph{Collaborations with experimental neuroscientists:} We will
collaborate with experimental neuroscientists at the Sainsbury Wellcome Centre
and at the Allen Institute for Neural Dynamics on the integration of machine
learning functionality into their experiments.

\paragraph{Collaborations with methods developers;} Recently Prof.~Garrett
Stanley contacted us asking for assistance in integrating into Bonsai.ML
functionality for close-loop neural control that his laboratory had developed
in RTXI/C++\footnote[6]{\url{https://stanley.gatech.edu/}}. We will assist his
laboratory on this integration.

Bonsai is already part of ONIX system for long-duration, low-latency and
high-throughput neural data
acquisition\footnote[7]{\url{https://www.nature.com/articles/s41592-024-02521-1}}.  We
will collaborate with the ONIX development team integrating Bonsai.ML
functionality into their system.

\subsection*{Governance}

We will create a Bonsai.ML governance structure comprising top-notch
neuroscientists that use Bonsai in their research and are heavily invested on
its future development.
%
This committee will advise us on building a long-term development roadmap for
Bonsai.ML.
