The proposed documentation and dissemination activities will help the users of
Bonsai.ML better understand the ML methods in the package, perform more
sophisticated neuroscience experiments, and produce unprecedented new
neuroscientific findings.
%
In addition, these activities will attract to Bonsai new experimental
neuroscience users interested in adding ML functionality to their experiments,
and create a new Bonsai community of machine learning methods developers, as we
explain below.
%
We have focused this proposal on applications of Bonsai.ML to experimental
neuroscience, since this is our area of expertise. However, Bonsai is used in
other experimental domains, like live exhibitions and robotics, where Bonsai.ML
should also be relevant.

Bonsai has demonstrated that providing experimental neuroscientists easy to use
tools for experimental control, allows them to create very sophisticated
experiments.
%
Bonsai.ML puts ML tools in the hands of experimental neuroscientists. With
these new tools and documentation, the level of sophistication of their
experiments, and the research findings that they produce, should greatly increase.

Most current ML methods are designed to operate offline, with datasets stored
on disk, after data collection has finished.
%
Bonsai requires ML methods that can process online data, while data is being
collected, and in a close-loop manner.
%
This has two important implications.
%
First, Bonsai.ML provides a new type of ML methods for real-time neuroscience
data, operating in in close loop. Used by experimental neuroscientists, these
method could generate unprecedented findings on brain function.
%
Second, Bonsai, as an excellent source of real-time neural and behavioural data,
could become of interest to ML methods developers wanting to apply their
methods to real-time data in close loop.
%
Therefore, the dissemination of Bonsai.ML among machine learning methods
developers could attract to Bonsai a new community of ML methods developers.
